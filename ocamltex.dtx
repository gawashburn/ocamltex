% \iffalse meta-comment
%
% Copyright (C) 2006 by Geoffrey Washburn <washburn@acm.org>
% Copyright (C) 2004 by Scott Pakin <scott+pt@pakin.org>
% ------------------------------------------------------
%
% This file may be distributed and/or modified under the
% conditions of the LaTeX Project Public License, either version 1.2
% of this license or (at your option) any later version.
% The latest version of this license is in:
%
%    http://www.latex-project.org/lppl.txt
%
% and version 1.2 or later is part of all distributions of LaTeX
% version 1999/12/01 or later.
%
% \fi
%
% \iffalse
%<*driver>
\ProvidesFile{ocamltex.dtx}
%</driver>
%<package>\NeedsTeXFormat{LaTeX2e}[1999/12/01]
%<package>\ProvidesPackage{ocamltex}
%<*package>
    [2006/07/02 v0.6 LaTeX macros for use with OCamlTeX]
%</package>
%
%<*driver>
\documentclass{ltxdoc}
\makeatletter
  \def\mlmac@pid{0}
  \def\mlmac@tag{FakingBeingRunFromOCamlTeX}
  \def\mlmac@tofile{\jobname.toml}
  \def\mlmac@fromfile{\jobname.frml}
  \def\mlmac@toflag{\jobname.tfml}
  \def\mlmac@fromflag{\jobname.ffml}
  \def\mlmac@doneflag{\jobname.dfml}
\makeatother
\usepackage{ocamltex}
\usepackage{xspace}
\usepackage{varioref}
\usepackage{flafter}
\usepackage{textcomp}
\usepackage{palatino}
\usepackage{euler}
\IfFileExists{hyperref.sty}{%
  \usepackage{hyperref}
  \hypersetup{%
    hyperindex=false,
    bookmarksopen,
    pdftitle={OCamlTeX: defining LaTeX macros in terms of OCaml code},
    pdfauthor={Geoffrey Washburn, washburn@acm.org},
    pdfsubject={Using OCaml to define LaTeX macros},
    pdfkeywords={programming, LaTeX, macros, OCaml}
  }
}{}
\EnableCrossrefs
\CodelineIndex
\RecordChanges
\setcounter{IndexColumns}{2}
\begin{document}
  \DocInput{ocamltex.dtx}
\end{document}
%</driver>
% \fi
%
% ^^A \CheckSum{468}
%
% \CharacterTable
%  {Upper-case    \A\B\C\D\E\F\G\H\I\J\K\L\M\N\O\P\Q\R\S\T\U\V\W\X\Y\Z
%   Lower-case    \a\b\c\d\e\f\g\h\i\j\k\l\m\n\o\p\q\r\s\t\u\v\w\x\y\z
%   Digits        \0\1\2\3\4\5\6\7\8\9
%   Exclamation   \!     Double quote  \"     Hash (number) \#
%   Dollar        \$     Percent       \%     Ampersand     \&
%   Acute accent  \'     Left paren    \(     Right paren   \)
%   Asterisk      \*     Plus          \+     Comma         \,
%   Minus         \-     Point         \.     Solidus       \/
%   Colon         \:     Semicolon     \;     Less than     \<
%   Equals        \=     Greater than  \>     Question mark \?
%   Commercial at \@     Left bracket  \[     Backslash     \\
%   Right bracket \]     Circumflex    \^     Underscore    \_
%   Grave accent  \`     Left brace    \{     Vertical bar  \|
%   Right brace   \}     Tilde         \~}
%
%
% \changes{v0.5}{2006/05/09}{Initial public release}
% \changes{v0.6}{2004/05/21}{Switched to .dtx packaging}
%
% \GetFileInfo{ocamltex.sty}
%
% \DoNotIndex{\$, \', \(, \), \., \\, \{, \}, \^, \", \@}
% \DoNotIndex{\@ifnextchar, \@ifundefined, \@makeother}
% \DoNotIndex{\b, \begingroup, \btrapped}
% \DoNotIndex{\d, \def}
% \DoNotIndex{\edef, \else, \endgroup, \expandafter}
% \DoNotIndex{\fi}
% \DoNotIndex{\gdef, \global}
% \DoNotIndex{\ifeof, \ifnum, \ifx, \immediate}
% \DoNotIndex{\let}
% \DoNotIndex{\makeatletter, \makeatother, \MessageBreak}
% \DoNotIndex{\n, \newcount, \newif, \newread, \newtoks, \newwrite}
% \DoNotIndex{\noexpand, \number}
% \DoNotIndex{\protect}
% \DoNotIndex{\r, \relax}
% \DoNotIndex{\s, \space, \string}
% \DoNotIndex{\the}
% \DoNotIndex{\W}
%
% \newcommand{\osn}[1]{\oldstylenums{#1}} 
%
% \newcommand{\OCamlTeX}{OCaml\TeX{}\xspace}           ^^A  Name of the OCamlTeX system
% \newcommand{\PerlTeX}{Perl\TeX{}\xspace}             ^^A  Name of the PerlTeX system
% \newcommand{\ocamltex}{\texttt{ocamltex.ml}\xspace}  ^^A  Name of the OCaml script
% \newcommand{\ocamlmac}{\texttt{ocamltex.sty}\xspace} ^^A  Name of the LaTeX package
%
% ^^A  Define an environment just like macro but for OCaml variables.
% \makeatletter
% \newenvironment{ocamlmacro}[1]{^^A
%   \begingroup
%     \def\PrintMacroName##1{\strut\MacroFont\string##1\ }^^A
%     \def\SpecialIndex@##1##2{^^A
%       \@bsphack
%       \special@index{\expandafter\@gobble\string##1\actualchar
%         \string\verb\quotechar*\verbatimchar\string##1\verbatimchar##2}^^A
%       \@esphack
%     }^^A
%     \begin{macro}{#1}^^A
% }{^^A
%     \end{macro}^^A
%   \endgroup
% }
% \makeatother
%
% \title{\OCamlTeX---defining {\LaTeX} macros in terms of OCaml code^^A
%   \thanks{This document corresponds to \OCamlTeX~\fileversion, dated
%   \filedate.}}
% \author{Geoffrey Washburn \\ \texttt{washburn@acm.org}}
%
% \maketitle
% \sloppy
%
% \begin{abstract}
%   \OCamlTeX{} is a combination OCaml script (\ocamltex) and \LaTeXe{}
%   style file (\ocamlmac) that, together, give the user the ability to
%   define \LaTeX{} macros in terms of OCaml code.  Once defined, a OCaml
%   macro becomes indistinguishable from any other \LaTeX{} macro.
%   \OCamlTeX{} thereby combines \LaTeX's typesetting power with OCaml's
%   programmability.
% \end{abstract}
%
% \section{Introduction}
%
% \TeX{} is a professional-quality typesetting system.  However, its
% programming language is quite difficult to master and use for anything but
% the most simple forms of text substitution.  Even \LaTeX, the most popular
% macro package for \TeX, does little to simplify \TeX{} programming.
% 
% OCaml is a general-purpose programming language, designed with program safety
% and reliability in mind~\cite{OCaml}.  It is very expressive, yet easy to
% learn and use.  Caml supports functional, imperative, and object-oriented
% programming styles.  It has been developed and distributed since \osn{1985}
% by \textsc{inria}, France's national research institute for computer science.
% OCaml does not have any direct support for typesetting itself.  However,
% Achim Blumensath has reimplemented many of the algorithms found in \TeX\ and
% code for manipulating related file formats in the Ant system~\cite{Ant}.
%
% \OCamlTeX's goal is to bridge these two worlds.  It enables the
% construction of documents that are primarily \LaTeX-based but contain
% a modicum of OCaml.  \OCamlTeX seamlessly integrates OCaml code into a
% \LaTeX{} document, enabling the user to define macros whose bodies
% consist of OCaml expressions instead of \TeX{} and \LaTeX{} code.
%
% As an example, suppose you need to define a macro that reverses a set
% of words.  Although it sounds like it should be simple, few \LaTeX{}
% authors are sufficiently versed in the \TeX{} language to be able to
% express such a macro.  However, a word-reversal function is easy to
% express in OCaml: one need only |split| a string into a list of words,
% |rev| the list, and |concat| it back together.  The following is how
% a |\reversewords| macro could be defined using \OCamlTeX:
%
% \begin{verbatim}
%   \ocamlnewcommand{\reversewords}[words]{
%     String.concat " "(List.rev (Str.split (Str.regexp " ") words))
%   }
% \end{verbatim}
%
% \noindent
% Then, executing ``\texttt{\string\reversewords\string{Try doing this without
% OCaml!\string}}'' in a document would produce the text ``OCaml!  without this
% doing Try''.  Simple, no?
% 
% As another example, think about how you would write a macro in \LaTeX{} to
% extract a substring of a given string when provided with a starting
% position and a length.  OCaml has an built-in |sub| function and
% \OCamlTeX{} makes it easy to export this to \LaTeX:
% ^^A
% \begin{verbatim}
%   \ocamlnewcommand{\substr}[str,strt,len]{
%     String.sub str (int_of_string strt) (int_of_string len)
%   }
% \end{verbatim}
% ^^A
% |\substr| can then be used just like any other \LaTeX{} macro---and as
% simply as OCaml's |String.sub| function:
%
% \bigskip
%
% \noindent|   |^^A
% \begin{tabular}{@{}l@{}}
%    |\newcommand{\str}{uniformly}| \\
%    |A sample substring of ``\str'' is ``\str'' is| \\
%    |``\substr{\str}{6}{2}''.| \\[1ex]
%    \multicolumn{1}{@{}c@{}}{\Huge$\Downarrow$} \\[2ex]
%    \multicolumn{1}{@{}c@{}}{A sample substring of ``uniformly'' is ``ml''.}
% \end{tabular}
%
% \bigskip
%
% To present a somewhat more complex example, observe how much
% easier it is to generate a repetitive matrix using OCaml code than
% ordinary \LaTeX{} commands:
%
% \begin{verbatim}
%   \ocamlnewcommand{\hilbertmatrix}[size]{
%    let entry i j = 
%        (if (i + j) > 0 then 
%           "\\frac{1}{" ^ (string_of_int (i + j + 1)) ^ "}"
%         else "1") in
%    let rec genRow i j len = 
%        (if (len = 0) then [] 
%         else (entry i j)::(genRow (i+1) j (len-1))) in
%    let rec genMatrix j mSize = 
%        (if (j = mSize) then []
%         else
%          ((String.concat " & " (genRow 0 j mSize)) ^ "\\\\ \n")::
%            (genMatrix (j+1) mSize)) in
%    let sizeInt = (int_of_string size) in       
%      ("\\[\n\\renewcommand{\\arraystretch}{1.3}\n") ^
%      ("\\begin{array}{") ^ (String.make sizeInt 'c') ^ ("}\n") ^
%      (String.concat "\n" (genMatrix 0 sizeInt)) ^ 
%      ("\\end{array}\n\\]")
%   }
%
%    \hilbertmatrix{15}
% \end{verbatim}
%
% \begin{center}
% {\Huge$\Downarrow$}
% \end{center}
%
% \[
% \renewcommand{\arraystretch}{1.3}
% \begin{array}{ccccccccccccccc}
% 1 & \frac{1}{2} & \frac{1}{3} & \frac{1}{4} & \frac{1}{5} & \frac{1}{6} & \frac{1}{7} & \frac{1}{8} & \frac{1}{9} & \frac{1}{10} & \frac{1}{11} & \frac{1}{12} & \frac{1}{13} & \frac{1}{14} & \frac{1}{15}\\ 
% \frac{1}{2} & \frac{1}{3} & \frac{1}{4} & \frac{1}{5} & \frac{1}{6} & \frac{1}{7} & \frac{1}{8} & \frac{1}{9} & \frac{1}{10} & \frac{1}{11} & \frac{1}{12} & \frac{1}{13} & \frac{1}{14} & \frac{1}{15} & \frac{1}{16}\\ 
% \frac{1}{3} & \frac{1}{4} & \frac{1}{5} & \frac{1}{6} & \frac{1}{7} & \frac{1}{8} & \frac{1}{9} & \frac{1}{10} & \frac{1}{11} & \frac{1}{12} & \frac{1}{13} & \frac{1}{14} & \frac{1}{15} & \frac{1}{16} & \frac{1}{17}\\ 
% \frac{1}{4} & \frac{1}{5} & \frac{1}{6} & \frac{1}{7} & \frac{1}{8} & \frac{1}{9} & \frac{1}{10} & \frac{1}{11} & \frac{1}{12} & \frac{1}{13} & \frac{1}{14} & \frac{1}{15} & \frac{1}{16} & \frac{1}{17} & \frac{1}{18}\\ 
% \frac{1}{5} & \frac{1}{6} & \frac{1}{7} & \frac{1}{8} & \frac{1}{9} & \frac{1}{10} & \frac{1}{11} & \frac{1}{12} & \frac{1}{13} & \frac{1}{14} & \frac{1}{15} & \frac{1}{16} & \frac{1}{17} & \frac{1}{18} & \frac{1}{19}\\ 
% \frac{1}{6} & \frac{1}{7} & \frac{1}{8} & \frac{1}{9} & \frac{1}{10} & \frac{1}{11} & \frac{1}{12} & \frac{1}{13} & \frac{1}{14} & \frac{1}{15} & \frac{1}{16} & \frac{1}{17} & \frac{1}{18} & \frac{1}{19} & \frac{1}{20}\\ 
% \frac{1}{7} & \frac{1}{8} & \frac{1}{9} & \frac{1}{10} & \frac{1}{11} & \frac{1}{12} & \frac{1}{13} & \frac{1}{14} & \frac{1}{15} & \frac{1}{16} & \frac{1}{17} & \frac{1}{18} & \frac{1}{19} & \frac{1}{20} & \frac{1}{21}\\ 
% \frac{1}{8} & \frac{1}{9} & \frac{1}{10} & \frac{1}{11} & \frac{1}{12} & \frac{1}{13} & \frac{1}{14} & \frac{1}{15} & \frac{1}{16} & \frac{1}{17} & \frac{1}{18} & \frac{1}{19} & \frac{1}{20} & \frac{1}{21} & \frac{1}{22}\\ 
% \frac{1}{9} & \frac{1}{10} & \frac{1}{11} & \frac{1}{12} & \frac{1}{13} & \frac{1}{14} & \frac{1}{15} & \frac{1}{16} & \frac{1}{17} & \frac{1}{18} & \frac{1}{19} & \frac{1}{20} & \frac{1}{21} & \frac{1}{22} & \frac{1}{23}\\ 
% \frac{1}{10} & \frac{1}{11} & \frac{1}{12} & \frac{1}{13} & \frac{1}{14} & \frac{1}{15} & \frac{1}{16} & \frac{1}{17} & \frac{1}{18} & \frac{1}{19} & \frac{1}{20} & \frac{1}{21} & \frac{1}{22} & \frac{1}{23} & \frac{1}{24}\\ 
% \frac{1}{11} & \frac{1}{12} & \frac{1}{13} & \frac{1}{14} & \frac{1}{15} & \frac{1}{16} & \frac{1}{17} & \frac{1}{18} & \frac{1}{19} & \frac{1}{20} & \frac{1}{21} & \frac{1}{22} & \frac{1}{23} & \frac{1}{24} & \frac{1}{25}\\ 
% \frac{1}{12} & \frac{1}{13} & \frac{1}{14} & \frac{1}{15} & \frac{1}{16} & \frac{1}{17} & \frac{1}{18} & \frac{1}{19} & \frac{1}{20} & \frac{1}{21} & \frac{1}{22} & \frac{1}{23} & \frac{1}{24} & \frac{1}{25} & \frac{1}{26}\\ 
% \frac{1}{13} & \frac{1}{14} & \frac{1}{15} & \frac{1}{16} & \frac{1}{17} & \frac{1}{18} & \frac{1}{19} & \frac{1}{20} & \frac{1}{21} & \frac{1}{22} & \frac{1}{23} & \frac{1}{24} & \frac{1}{25} & \frac{1}{26} & \frac{1}{27}\\ 
% \frac{1}{14} & \frac{1}{15} & \frac{1}{16} & \frac{1}{17} & \frac{1}{18} & \frac{1}{19} & \frac{1}{20} & \frac{1}{21} & \frac{1}{22} & \frac{1}{23} & \frac{1}{24} & \frac{1}{25} & \frac{1}{26} & \frac{1}{27} & \frac{1}{28}\\ 
% \frac{1}{15} & \frac{1}{16} & \frac{1}{17} & \frac{1}{18} & \frac{1}{19} & \frac{1}{20} & \frac{1}{21} & \frac{1}{22} & \frac{1}{23} & \frac{1}{24} & \frac{1}{25} & \frac{1}{26} & \frac{1}{27} & \frac{1}{28} & \frac{1}{29}\\ 
% \end{array}
% \]
%
% \bigskip
%
% In addition to |\ocamlnewcommand| and |\ocamlrenewcommand|, \OCamlTeX{}
% supports |\ocamlnewenvironment| and |\ocamlrenewenvironment| macros.
% These enable environments to be defined using OCaml code.  The
% following example, a |spreadsheet| environment, generates a |tabular|
% environment plus a predefined header row.  This example would have
% been much more difficult to implement without \OCamlTeX:
%
% \begin{verbatim}
% \newcounter{ssrow}
% \perlnewenvironment{spreadsheet}[1]{
%   my $cols = $_[0];
%   my $header = "A";
%   my $tabular = "\\setcounter{ssrow}{1}\n";
%   $tabular .= '\newcommand*{\rownum}{\thessrow\addtocounter{ssrow}{1}}' . "\n";
%   $tabular .= '\begin{tabular}{@{}r|*{' . $cols . '}{r}@{}}' . "\n";
%   $tabular .= '\\multicolumn{1}{@{}c}{} &' . "\n";
%   foreach (1 .. $cols) {
%     $tabular .= "\\multicolumn{1}{c";
%     $tabular .= '@{}' if $_ == $cols;
%     $tabular .= "}{" . $header++ . "}";
%     if ($_ == $cols) {
%       $tabular .= " \\\\ \\cline{2-" . ($cols+1) . "}"
%     }
%     else {
%       $tabular .= " &";
%     }
%     $tabular .= "\n";
%   }
%   return $tabular;
% }{
%   return "\\end{tabular}\n";
% }
%
% \begin{center}
%   \begin{spreadsheet}{4}
%     \rownum &  1 &  8 & 10 & 15 \\
%     \rownum & 12 & 13 &  3 &  6 \\
%     \rownum &  7 &  2 & 16 &  9 \\
%     \rownum & 14 & 11 &  5 &  4
%   \end{spreadsheet}
% \end{center}
% \end{verbatim}
%
% \begin{center}
% {\Huge$\Downarrow$}
% \end{center}
%
% \DeleteShortVerb{\|}
% \begin{center}
%   \begin{tabular}{@{}r|*{4}{r}@{}}
%     \multicolumn{1}{@{}c}{} &
%     \multicolumn{1}{c}{A} &
%     \multicolumn{1}{c}{B} &
%     \multicolumn{1}{c}{C} &
%     \multicolumn{1}{c@{}}{D} \\ \cline{2-5}
%     1 &  1 &  8 & 10 & 15 \\
%     2 & 12 & 13 &  3 &  6 \\
%     3 &  7 &  2 & 16 &  9 \\
%     4 & 14 & 11 &  5 &  4
%   \end{tabular}
% \end{center}
% \MakeShortVerb{\|}
%
%
% \section{Usage}
%
% There are two components to using \OCamlTeX\@.  First, documents must
% include a ``|\usepackage{ocamltex}|'' line in their preamble in
% order to define |\ocamlnewcommand|, |\ocamlrenewcommand|,
% |\ocamlnewenvironment|, |\ocamlrenewenvironment|, and |\ocamlexec|.  
% Second, \LaTeX{} documents must be compiled using the \ocamltex{} wrapper script.
%
% \subsection{Defining and redefining OCaml macros}
%
% \DescribeMacro{\ocamlnewcommand}
% \DescribeMacro{\ocamlrenewcommand}
% \DescribeMacro{\ocamlnewenvironment}
% \DescribeMacro{\ocamlrenewenvironment}
% \DescribeMacro{\ocamlexec}
% \ocamlmac{} defines five macros: |\ocamlnewcommand|,
% |\ocamlrenewcommand|, |\ocamlnewenvironment|, 
% |\ocamlrenewenvironment|, and |\ocamlexec|.  The first four behave similarly
% to their \LaTeXe{} counterparts---|\newcommand|, |\renewcommand|, 
% |\newenvironment|, and |\renewenvironment|.  The two primary differences are 
% \begin{itemize} 
% \item the macro body consists of an OCaml expression that generates an OCaml string containing \LaTeX{} code,
% \item instead of simply specifying the number of macro arguments, comma separated names
% are used instead.
% \end{itemize}
% ^^A \ocamlmac{} even includes
% ^^A support for optional arguments and the starred forms of its commands
% ^^A (i.e.~|\perlnewcommand*|, |\perlrenewcommand*|, |\perlnewenvironment*|,
% ^^A and |\perlrenewenvironment*|).
%
% When the OCaml expression is evaluated, it is placed within a function named
% after the macro name but with ``|\|'' replaced with ``|latex_|''.  For
% example, a \OCamlTeX-defined \LaTeX{} macro called |\myMacro| produces
% an OCaml function called |latex_myMacro|.  The named macro arguments are
% converted to OCaml function arguments of type |string|.  
%
% Any valid OCaml expressions can be used in the body of a macro.  However, 
% because there are some languages features that cannot be used inside of expressions,
% we provide another macro, |\ocamlexec|, that can be used to execute OCaml top-level
% commands.  For example, many interesting uses of OCaml require the definition of new
% data types.  If we wanted to define a new type for binary trees for use by later macros,
% we might use the following \LaTeX\ code.
% ^^A
% \begin{verbatim}
% \ocamlexec{type mytree = Leaf | Node of mytree * mytree}
% \end{verbatim}
% ^^A
% The same OCaml environment is used for the entire |latex| run, so unless a
% data type definition is shadowed, it will remain in scope for the remainder
% of the program.  Therefore macros defined by |\ocamlnewcommand| can invoke
% each other.  It also means that reference cells persist across macro calls:
%
% \bigskip
%
% \noindent|   |^^A
% \begin{tabular}{@{}l@{}}
%    |\ocamlexec{val x = ref ""}| \\
%    |\ocamlnewcommand{\setX}[arg]{x := arg; "\relax"}| \\
%    |\ocamlnewcommand{\getX}{"$x$ was set to " ^ (!x) ^ "."}| \\
%    |\setX{123}| \\
%    |\getX| \\
%    |\setX{456}| \\
%    |\getX| \\[1ex]
%    \multicolumn{1}{@{}c@{}}{\Huge$\Downarrow$} \\[2ex]
%    \multicolumn{1}{@{}c@{}}{$x$ was set to \osn{123}.  $x$ was set to \osn{456}.} \\
% \end{tabular}
%
% \bigskip
%
% Macro arguments are expanded by \LaTeX{} before being passed to OCaml.
% Consider the following macro definition, which wraps its argument within
% |\begin{verbatim*}|\dots\linebreak[0]|\end{verbatim*}|:
%
% \begin{verbatim}
%     \ocamlnewcommand{\verbit}[arg]{
%       "\\begin{verbatim*}\n" ^ arg ^ "\n\\end{verbatim*}\n"
%     }
% \end{verbatim}
%
% \noindent
% An invocation of
% ``|\verbit{\TeX}|'' would therefore typeset the \emph{expansion} of
% ``|\TeX|'', namely ``|T\kern| |-.1667em\lower| |.5ex\hbox| |{E}\kern|
% |-.125emX\spacefactor| |\@m|'', which might be a bit unexpected.  The
% solution is to use |\noexpand|:
% |\verbit{\noexpand\TeX}|~$\Rightarrow$ |\TeX|\@.  ``Robust'' macros as
% well as |\begin| and |\end| are implicitly preceded by |\noexpand|.
%
%
% \subsection{Invoking \ocamltex}
% \label{sec:ocamltex-man}
%
% The following pages reproduce the \ocamltex program documentation.
% Key parts of the documentation are excerpted when \ocamltex is
% invoked with the |--help| option. 
%
% \clearpage
% \begingroup
% \def\index#1{}
% \def\section#1{}
% \let\subsection=\subsubsection
%
% \def\C++{{\rm C\kern-.05em\raise.3ex\hbox{\footnotesize ++}}}
% \def\underscore{\leavevmode\kern.04em\vbox{\hrule width 0.4em height 0.3pt}}
% \setlength{\parindent}{0pt}
%
% \section{OCAMLTEX.ML}%
% \index{OCAMLTEX.ML}
%
% \subsection*{NAME}
% ocamltex --- enable \LaTeX{} macros to be defined in terms of OCaml code
%
% \subsection*{SYNOPSIS}
% ocamltex
% {\tt [}{\bf {\tt --}help}{\tt ]}
% {\tt [}{\bf {\tt --}latex}={\em program\/}{\tt ]}
% {\tt [}{\bf {\tt --}}{\tt [}{\bf no}{\tt ]}{\bf safe}{\tt ]}
% {\tt [}{\bf {\tt --}permit}={\em feature\/}{\tt ]}
% {\tt [}{\em latex options\/}{\tt ]}
%
% \subsection*{DESCRIPTION}
% \LaTeX{} --- through the underlying \TeX{} typesetting system --- produces
% beautifully typeset documents but has a macro language that is
% difficult to program.  
%
% \bigskip
%
% Clearly, Perl's programmability could complement \LaTeX{}'s typesetting
% strengths.  {\bf perltex} is the tool that enables a symbiosis between
% the two systems.  All a user needs to do is compile a \LaTeX{} document
% using {\bf perltex} instead of {\bf latex}.  ({\bf perltex} is actually a
% wrapper for {\bf latex}, so no {\bf latex} functionality is lost.)  If the
% document includes a {\tt \char`\\usepackage\{perltex\}} in its preamble, then
% {\tt \char`\\perlnewcommand} and {\tt \char`\\perlrenewcommand} macros will be made
% available.  These behave just like \LaTeX{}'s {\tt \char`\\newcommand} and
% {\tt \char`\\renewcommand} except that the macro body contains Perl code instead
% of \LaTeX{} code.
%
% \subsection*{OPTIONS}%
% \index{OPTIONS}
%
% {\bf perltex} accepts the following command-line options:
%
% \begin{description}
%
% \item[{\bf {\tt -}h $\mid$ {\tt --}h $\mid$ {\tt -}help $\mid$ {\tt --}help}]%
% \index{--help@{\bf {\tt --}help}}%
% \hfil\\
% Display basic usage information.
%
% \item[{\bf {\tt --}latex}={\em program\/}]%
% \index{--latex=program@{\bf {\tt --}latex}={\em program\/}}%
% \hfil\\
% Specify a program to use instead of {\bf latex}.  For example,
% {\tt --latex=pdf{}latex} would typeset the given document using
% {\bf pdf{}latex} instead of ordinary {\bf latex}.
%
% \end{description}
%
% These options are then followed by whatever options are normally
% passed to {\bf latex} (or whatever program was specified with
% {\tt --latex}), including, for instance, the name of the {\em .tex\/} file to
% compile.
%
% \subsection*{EXAMPLES}
% In its simplest form, {\bf ocamltex} is run just like {\bf latex}:
% \begin{verbatim}
%     ocamltex myfile.tex
% \end{verbatim}
%
% To use {\bf pdf{}latex} instead of regular {\bf latex}, use the {\tt --latex}
% option:
% \begin{verbatim}
%     ocamltex --latex=pdflatex myfile.tex
% \end{verbatim}
%
% \subsection*{ENVIRONMENT}
% {\bf ocamltex} honors the following environment variables:
%
% \begin{description}
%
% \item[OCAMLTEX]%
% \index{OCAMLTEX@OCAMLTEX}%
% \hfil\\
% Specify the filename of the \LaTeX{} compiler.  The \LaTeX{} compiler
% defaults to ``{\tt latex}''.  The {\tt OCAMLTEX} environment variable
% overrides this default, and the {\tt --latex} command-line option (see
% the {\tt OPTIONS} entry elsewhere in this document) overrides that.
%
% \end{description}
%
% \subsection*{FILES}
% While compiling {\em jobname.tex\/}, {\bf ocamltex} makes use of the following
% files:
%
% \begin{description}
%
% \item[{\em jobname.lgml\/}]%
% \index{jobname.lgml@{\em jobname.lgml\/}}%
% \hfil\\
% log file written by OCaml; helpful for debugging OCaml macros
%
% \item[{\em jobname.toml\/}]%
% \index{jobname.toml@{\em jobname.toml\/}}%
% \hfil\\
% information sent from \LaTeX{} to OCaml
%
% \item[{\em jobname.frml\/}]%
% \index{jobname.frml@{\em jobname.frml\/}}%
% \hfil\\
% information sent from OCaml to \LaTeX{}
%
% \item[{\em jobname.tfml\/}]%
% \index{jobname.tfml@{\em jobname.tfml\/}}%
% \hfil\\
% ``flag'' file whose existence indicates that {\em jobname.toml\/} contains
% valid data
%
% \item[{\em jobname.ffml\/}]%
% \index{jobname.ffml@{\em jobname.ffml\/}}%
% \hfil\\
% ``flag'' file whose existence indicates that {\em jobname.frml\/} contains
% valid data
%
% \item[{\em jobname.dfml\/}]%
% \index{jobname.dfpl@{\em jobname.dfml\/}}%
% \hfil\\
% ``flag'' file whose existence indicates that {\em jobname.ffml\/} has been
% deleted
%
% \end{description}
%
% \subsection*{SEE ALSO}
% {\em latex\/}(1), {\em pdf{}latex\/}(1), {\em ocaml\/}(1),
%
% \subsection*{AUTHORS}
% Geoffrey Washburn, {\em washburn@acm.org\/} \\
% Scott Pakin, {\em scott+pt@pakin.org\/}
%
% \endgroup
% \clearpage
%
%
% \StopEventually{^^A
%   \section{License agreement}
%   \label{sec:license}
%
%   \begin{center}
%   Copyright \textcopyright{} 2006 by Geoffrey Washburn \texttt{<washburn@acm.org>} \\
%   Copyright \textcopyright{} 2004 by Scott Pakin \texttt{<scott+pt@pakin.org>}
%   \end{center}
%
%   \providecommand{\url}[1]{\texttt{##1}}
%
%   \noindent
%   These files may be distributed and/or modified under the conditions of
%   the \LaTeX{} Project Public License, either version~1.2 of this
%   license or (at your option) any later version.  The latest version of
%   this license is in \url{http://www.latex-project.org/lppl.txt} and
%   version~1.2 or later is part of all distributions of \LaTeX{} version
%   1999/12/01 or later.
%
%   \PrintChanges
%   \PrintIndex
% }
%
% \section{Implementation}
% \label{sec:implementation}
%
% Users interested only in \emph{using} \OCamlTeX can skip
% Section~\ref{sec:implementation}, which presents the complete
% \OCamlTeX source code.  This section should be of interest primarily
% to those who wish to extend \OCamlTeX or modify it to use a language
% other than OCaml.
%
% Section~\ref{sec:implementation} is split into two main parts.
% Section~\ref{sec:ocamlmacros} presents the source code for \ocamlmac,
% the \LaTeX{} side of \OCamlTeX, and Section~\ref{sec:ocaml-env} presents
% the source code for \ocamltex, the OCaml side of \OCamlTeX\@.
% \makeatletter
% \@ifundefined{ocamlmaclines}{}{^^A
%   \newcount\ocamltex@lines
%   \ocamltex@lines=\ocamltexlines
%   \advance\ocamltex@lines by -\ocamlmaclines
%   In toto, \OCamlTeX{} consists of a relatively small amount of code.
%   \ocamlmac is only \ocamlmaclines{} lines of \LaTeX{} and \ocamltex
%   is only \the\ocamltex@lines{} lines of OCaml.
% }
% \makeatother
% \ocamltex is fairly straightforward OCaml code and should not be too difficult
% to understand by anyone comfortable with OCaml programming.  \ocamlmac, in
% contrast, contains a bit of {\LaTeX} trickery and is probably impenetrable to
% anyone who hasn't already tried her hand at {\LaTeX} programming.
% Fortunately for the reader, the code is profusely commented so the aspiring
% {\LaTeX} guru may yet learn from it.
%
% After documenting the \ocamlmac and \ocamltex source code, a few suggestions
% are provided for porting \OCamlTeX to use a backend language other than OCaml
% (Section~\ref{sec:porting}), differences from \PerlTeX
% (Section~\ref{sec:differences}), as well as information on future
% enchancements (Section~\ref{sec:future}.
%
% \subsection{\ocamlmac}
% \label{sec:ocamlmacros}
%
% \iffalse
%<*package>
% \fi
%
% The key \LaTeX\ wizardry required to make \ocamlmac possible includes: 
%
% \begin{enumerate}
%   \item storing brace-matched---but otherwise not valid \LaTeX---code
%   for later use
%
%   \item iterating over a macro's arguments
% \end{enumerate}
%
% Storing non-\LaTeX\ code in a variable involves beginning a group in
% an argument-less macro, fiddling with category codes, using
% |\afterassignment| to specify a continuation, and storing the
% subsequent brace-delimited tokens in the input stream into a token
% register.  The continuation, which also takes no arguments,
% ends the group begun in the first function and proceeds using the
% correctly |\catcode|d token register.  This technique appears in
% |\mlmac@haveargs| and |\mlmac@havecode| and in a simpler form
% (i.e.,~without the need for storing the argument) in
% |\mlmac@write@ocaml| and |\mlmac@write@ocaml@i|.
%
% Iterating over a macro's arguments is hindered by \TeX's requirement
% that ``|#|'' be followed by a number or another ``|#|''.  The
% technique used (which originates from the Texinfo source code) is
% first to |\let| a variable be |\relax|, thereby making it
% unexpandable, then to define a macro that uses that variable followed
% by a loop variable, and finally to expand the loop variable and |\let|
% the |\relax|ed variable be ``|#|'' right before invoking the macro.
% This technique appears in |\mlmac@havecode|.
%
% \subsubsection{Package initialization}
% \label{sec:package-init}
%
% \OCamlTeX defines six macros that are used for communication between
% OCaml and \LaTeX\@.  |\mllmac@tag| is a string of characters that should
% never occur within one of the user's macro names, macro arguments, or
% macro bodies.  \ocamltex therefore defines |\mlmac@tag| as a long
% string of random uppercase letters.  |\mlmac@tofile| is the name of a
% file used for communication from \LaTeX{} to OCaml.  |\mlmac@fromfile|
% is the name of a file used for communication from OCaml to \LaTeX.
% |\mlmac@toflag| signals that |\mlmac@tofile| can be read safely.
% |\mlmac@fromflag| signals that |\mlmac@fromfile| can be read safely.
% |\mlmac@doneflag| signals that |\mlmac@fromflag| has been deleted.
% Table~\ref{tbl:variables} lists all of these variables along with the
% value assigned to each by \ocamltex.
%
% \begin{table}[htbp]
%   \centering
%   \caption{Variables used for communication between OCaml and \LaTeX}
%   \label{tbl:variables}
%   \begin{tabular}{@{}lll@{}}
%     \hline
%     Variable & Purpose & \ocamltex assignment \\
%     \hline
%
%     |\mlmac@pid| & Used to identity stale files &
%     (\ocamltex process id) \\
%
%     |\mlmac@tag| & |\mlmac@tofile| field separator &
%     (20 random letters) \\
%
%     |\mlmac@tofile| & \LaTeX{} $\rightarrow$ OCaml communication &
%     |\jobname.toml| \\
%
%     |\mlmac@fromfile| & OCaml $\rightarrow$ \LaTeX{} communication &
%     |\jobname.frml| \\
%
%     |\mlmac@toflag| & |\mlmac@tofile| synching &
%     |\jobname.tfml| \\
%
%     |\mlmac@fromflag| & |\mlmac@fromfile| synching &
%     |\jobname.ffml| \\
%
%     |\mlmac@doneflag| & |\mlmac@fromflag| synching &
%     |\jobname.dfml| \\
%
%     \hline
%   \end{tabular}
% \end{table}
%
% \begin{macro}{\ifmlmac@have@ocamltex}
% \begin{macro}{\mlmac@have@ocamltextrue}
% \begin{macro}{\mlmac@have@ocamltexfalse}
% The following block of code checks the existence of each of the
% variables listed in Table~\ref{tbl:variables}.  If any variable is not
% defined, \ocamlmac gives an error message and---as we shall see on
% page~\pageref{page:define-dummies}---defines dummy versions of
% |\ocaml|[|re|]|newcommand|, |\ocaml|[|re|]|newenvironment|, and |\ocamlexec|.
%    \begin{macrocode}
\newif\ifmlmac@have@ocamltex
\mlmac@have@ocamltextrue
\@ifundefined{mlmac@pid}{\mlmac@have@ocamltexfalse}{}
\@ifundefined{mlmac@tag}{\mlmac@have@ocamltexfalse}{}
\@ifundefined{mlmac@tofile}{\mlmac@have@ocamltexfalse}{}
\@ifundefined{mlmac@fromfile}{\mlmac@have@ocamltexfalse}{}
\@ifundefined{mlmac@toflag}{\mlmac@have@ocamltexfalse}{}
\@ifundefined{mlmac@fromflag}{\mlmac@have@ocamltexfalse}{}
\@ifundefined{mlmac@doneflag}{\mlmac@have@ocamltexfalse}{}

% These are really hackish macros to calculate the length of a comma
% separated list of argument names.

\def\getlength@i#1{\ifx#1\end \let\next=\relax 
 \else
   \ifx#1\comma \advance\count0 by1 \let\next=\getlength@ii 
     \else \let\next=\getlength@ii\fi\fi \next}

\def\getlength@ii#1{\ifx#1\end \advance\count0 by1 \let\next=\relax 
 \else
   \ifx#1\comma \advance\count0 by1 \let\next=\getlength@ii 
     \else \let\next=\getlength@ii\fi\fi \next}

\ifmlmac@have@ocamltex
\else
  \PackageError{ocamltex}{Document must be compiled using ocamltex}
    {Instead of compiling your document directly with latex, you need
     to\MessageBreak use the ocamltex script.  \space ocamltex sets up
     a variety of macros needed by\MessageBreak the ocamltex
     package as well as a listener process needed for\MessageBreak
     communication between LaTeX and OCaml.}
\fi
%    \end{macrocode}
% \end{macro}
% \end{macro}
% \end{macro}
%
% \subsubsection{Defining Ocaml macros}
% \label{sec:ocaml-mac}
%
% \OCamlTeX defines four macros intended to be called by the user.
% Section~\ref{sec:ocaml-mac} details the implementation of two of them:
% |\ocamlnewcommand| and |\ocamlrenewcommand|.  (Section~\ref{sec:ocaml-env}
% details the implementation of the other two, |\ocamlnewenvironment| and
% |\ocamlrenewenvironment|.)  The goal is for these two macros to behave
% \emph{exactly} like |\newcommand| and |\renewcommand|, respectively, except
% that the macros they define have OCaml bodies instead of \LaTeX{} bodies and
% use a list of explicitly named arguments.
%
% The sequence of the operations defined in this section is as follows:
%
% \begin{enumerate}
%   \item The user invokes |\ocaml|[|re|]|newcommand|, which stores
%   |\|[|re|]|newcommand| in |\mlmac@command|.  The
%   |\ocaml|[|re|]|newcommand| macro then invokes |\mlmac@newcommand@i|
%   with a first argument of ``|*|'' for |\ocaml|[|re|]|newcommand*| or
%   ``|!|'' for ordinary |\ocaml|[|re|]|newcommand|.
%
%   \item |\mlmac@newcommand@i| defines |\mlmac@starchar| as ``|*|'' if
%   it was passed a ``|*|'' or \meta{empty} if it was passed a ``|!|''.
%   It then stores the name of the user's macro in |\mlmac@macname|, a
%   |\write|able version of the name in |\mlmac@cleaned@macname|, and
%   the macro's previous definition (needed by |\ocamlrenewcommand|) in
%   |\mlmac@oldbody|.  Finally, |\mlmac@newcommand@i| invokes
%   |\mlmac@newcommand@ii|.
%
%   \item |\mlmac@newcommand@ii| counts the number of comma separated arguments
%   to the user's macro (which may be zero) in |\mlmac@numargs|.  Additionally,
%   it stores the actual argument names in |\mlmac@argsstr|.  It then invokes
%   |\mlmac@newcommand@iii@opt| if the first argument is supposed to be
%   optional or |\mlmac@newcommand@iii@no@opt| if all arguments are supposed to
%   be required.
%
%   \item |\mlmac@newcommand@iii@opt| defines |\mlmac@defarg| as the
%   default value of the optional argument.
%   |\mlmac@newcommand@iii@no@opt| defines it as \meta{empty}.  Both
%   functions then call |\mlmac@haveargs|.
%
%   \item |\mlmac@haveargs| stores the user's macro body (written in
%   OCaml) verbatim in |\mlmac@ocamlcode|.  |\mlmac@haveargs| then invokes
%   |\mlmac@havecode|.
%
%   \item By the time |\mlmac@havecode| is invoked all of the information
%   needed to define the user's macro is available.  Before defining a \LaTeX{}
%   macro, however, |\mlmac@havecode| invokes |\mlmac@write@ocaml| to tell
%   \ocamltex to define a OCaml subroutine with a name based on
%   |\mlmac@cleaned@macname| and the code contained in |\mlmac@ocamlcode|.
%   Figure~\ref{fig:tofile-define} illustrates the data that
%   |\mlmac@write@perl| passes to \ocamltex.  Note that each line is implicitly
%   followed by a carriage-return.
%
% \begin{figure}[t]
%   \hrule
%   \vspace{0.5em}
%   \centering
%   \DeleteShortVerb\|
%   \begin{tabular}{|l|}
%     \hline
%     \verb|DEF|                    \\ \hline
%     \verb|\mlmac@tag|             \\ \hline
%     \verb|\mlmac@cleaned@macname| \\ \hline
%     \verb|\mlmac@tag|             \\ \hline
%     \verb|\mlmac@argsstr|         \\ \hline
%     \verb|\mlmac@tag|             \\ \hline
%     \verb|\mlmac@ocamlcode|       \\ \hline
%   \end{tabular}
%   \MakeShortVerb\|
%   \caption{Data written to \texttt{\string\mlmac@tofile} to define an OCaml function}
%   \label{fig:tofile-define}
%   \vspace{0.5em}
%   \hrule
% \end{figure}
%
%   \item |\mlmac@havecode| invokes |\newcommand| or |\renewcommand|, as
%   appropriate, defining the user's macro as a call to |\mlmac@write@ocaml|.
%   An invocation of the user's \LaTeX{} macro causes |\mlmac@write@ocaml| to
%   pass the information shown in Figure~\ref{fig:tofile-use} to \ocamltex.
%   Again, note that each line is implicitly followed by a carriage-return.
%
% \begin{figure}[t]
%   \hrule
%   \vspace{0.5em}
%   \centering
%   \DeleteShortVerb\|
%   \begin{tabular}{|l|}
%     \hline
%     \verb|USE|                    \\ \hline
%     \verb|\mlmac@tag|             \\ \hline
%     \verb|\mlmac@cleaned@macname| \\ \hline
%     \verb|\mlmac@tag|             \\ \hline
%     \verb|#1|                     \\ \hline
%     \verb|\mlmac@tag|             \\ \hline
%     \verb|#2|                     \\ \hline
%     \verb|\mlmac@tag|             \\ \hline
%     \verb|#3|                     \\ \hline
%     \multicolumn{1}{c}{$\vdots$}  \\ \hline
%     \verb|#|\meta{last}           \\ \hline
%   \end{tabular}
%   \MakeShortVerb\|
%   \caption{Data written to \texttt{\string\mlmac@tofile} to invoke a
%     OCaml function}
%   \label{fig:tofile-use}
%   \vspace{0.5em}
%   \hrule
% \end{figure}
%
%   \item Whenever |\mlmac@write@ocaml| is invoked it writes its argument
%   verbatim to |\mlmac@tofile|; \ocamltex evaluates the code and
%   writes |\mlmac@fromfile|; finally, |\mlmac@write@ocaml| |\input|s
%   |\mlmac@fromfile|.
% \end{enumerate}
%
% An example might help distinguish the myriad macros used internally by
% \ocamlmac.  Consider the following call made by the user's
% document:\label{text:example-cmd}
%
% \begin{center}
% |\ocamlnewcommand{\example}[fst,snd]{ fst ^ "---" ^ snd }|
% \end{center}
%
% \noindent
% Table~\ref{tbl:example-cmd} shows how \ocamlmac parses that command
% into its constituent components and which components are bound to
% which \ocamlmac macros.
%
% \begin{table}[htbp]
%   \centering
%   \caption{Macro assignments corresponding to an sample
%     \texttt{\string\ocamlnewcommand*}}
%   \label{tbl:example-cmd}
%   \begin{tabular}{@{}lll@{}}
%     \hline
%     Macro & \multicolumn{2}{l@{}}{Sample definition} \\
%     \hline
%     |\mlmac@command|          & |\newcommand|                    \\
%     |\mlmac@macname|          & |\example|                       \\
%     |\mlmac@cleaned@macname|  & |\example| & (catcode~11)        \\
%     |\mlmac@oldbody|          & |\relax| & (presumably)          \\
%     |\mlmac@numargs|          & |2|                              \\
%     |\mlmac@argsstr|          & |fst,snd|                        \\
%     |\mlmac@defarg|           & |frobozz|                        \\
%     |\mlmac@ocamlcode|         & |fst ^ "---" ^ snd| & (catcode~11) \\
%     \hline
%   \end{tabular}
% \end{table}
%
% \bigskip
%
% \begin{macro}{\ocamlnewcommand}
% \begin{macro}{\ocamlrenewcommand}
% \begin{macro}{\mlmac@command}
% \begin{macro}{\mlmac@next}
% |\ocamlnewcommand| and |\ocamlrenewcommand| are the first two commands
% exported to the user by \ocamlmac.  |\ocamlnewcommand| is analogous to
% |\newcommand| except that the macro body consists of OCaml code instead
% of \LaTeX{} code.  Likewise, |\ocamlrenewcommand| is analogous to
% |\renewcommand| except that the macro body consists of OCaml code
% instead of \LaTeX{} code.  |\ocamlnewcommand| and |\ocamlrenewcommand|
% merely define |\mlmac@command| and |\mlmac@next| and invoke
% |\mlmac@newcommand@i|.
%    \begin{macrocode}
\def\ocamlnewcommand{%
  \let\mlmac@command=\newcommand
  \let\mlmac@next=\relax
  % Disable optional argument support for now
  % \@ifnextchar*{\mlmac@newcommand@i}{\mlmac@newcommand@i!}%
  \mlmac@newcommand@i!%
}
%    \end{macrocode}
%    \begin{macrocode}
\def\ocamlrenewcommand{%
  \let\mlmac@next=\relax
  \let\mlmac@command=\renewcommand
  % Disable optional argument support for now
  % \@ifnextchar*{\mlmac@newcommand@i}{\mlmac@newcommand@i!}%
  \mlmac@newcommand@i!%
}
%    \end{macrocode}
% \end{macro}
% \end{macro}
% \end{macro}
% \end{macro}
%
% \begin{macro}{\mlmac@newcommand@i}
% \begin{macro}{\mlmac@starchar}
% \begin{macro}{\mlmac@macname}
% \begin{macro}{\mlmac@oldbody}
% \begin{macro}{\mlmac@cleaned@macname}
% If\label{page:newcommand-i} the user invoked
% |\ocaml|[|re|]|newcommand*| then |\mlmac@newcommand@i| is passed a
% ``|*|'' and, in turn, defines |\mlmac@starchar| as ``|*|''.  If the
% user invoked |\ocaml|[|re|]|newcommand| (no ``|*|'') then
% |\mlmac@newcommand@i| is passed a ``|!|'' and, in turn, defines
% |\mlmac@starchar| as \meta{empty}.  In either case,
% |\mlmac@newcommand@i| defines |\mlmac@macname| as the name of the
% user's macro, |\mlmac@cleaned@macname| as a |\write|able
% (i.e.,~category code~11) version of |\mlmac@macname|, and
% |\mlmac@oldbody| and the previous definition of the user's macro.
% (|\mlmac@oldbody| is needed by |\ocamlrenewcommand|.)  It then invokes
% |\mlmac@newcommand@ii|.
%    \begin{macrocode}
\def\mlmac@newcommand@i#1#2{%
  \ifx#1*%
    \def\mlmac@starchar{*}%
  \else
    \def\mlmac@starchar{}%
  \fi
  \def\mlmac@macname{#2}%
  \let\mlmac@oldbody=#2\relax
  \expandafter\def\expandafter\mlmac@cleaned@macname\expandafter{%
    \expandafter\string\mlmac@macname}%
  \@ifnextchar[{\mlmac@newcommand@ii}{\mlmac@newcommand@ii[]}%]
}
%    \end{macrocode}
% \end{macro}
% \end{macro}
% \end{macro}
% \end{macro}
% \end{macro}
%
% \begin{macro}{\mlmac@newcommand@ii}
% \begin{macro}{\mlmac@numargs}
% \begin{macro}{\mlmac@argsstr}
% |\mlmac@newcommand@i| invokes |\mlmac@newcommand@ii| with comma separated
% arguments to the user's macro in brackets.  |\mlmac@newcommand@ii| counts the
% number of arguments, stores that number in |\mlmac@numargs| and stores the
% comma separated arguments in |\mlmac@argsstr|, and invokes
% |\mlmac@newcommand@iii@opt| if the first argument is to be optional or
% |\mlmac@newcommand@iii@no@opt| if all arguments are to be mandatory.
%    \begin{macrocode}
\def\mlmac@newcommand@ii[#1]{%
  \begingroup
    \count0=0 %
    \let\comma=,% 
    \getlength@i#1\end %
    \xdef\mlmac@numargs{\number\count0}%
    \xdef\mlmac@argsstr{#1}%
  \endgroup
  \@ifnextchar[{\mlmac@newcommand@iii@opt}
               {\mlmac@newcommand@iii@no@opt}%]
}
%    \end{macrocode}
% \end{macro}
% \end{macro}
% \end{macro}
%
% \begin{macro}{\mlmac@newcommand@iii@opt}
% \begin{macro}{\mlmac@newcommand@iii@no@opt}
% \begin{macro}{\mlmac@defarg}
% Only one of these two macros is executed per invocation of
% |\ocaml|[|re|]|newcommand|, depending on whether or not the first
% argument of the user's macro is an optional argument.
% |\mlmac@newcommand@iii@opt| is invoked if the argument is optional.
% It defines |\mlmac@defarg| to the default value of the optional
% argument.  |\mlmac@newcommand@iii@no@opt| is invoked if all arguments
% are mandatory.  It defines |\mlmac@defarg| as |\relax|.  Both
% |\mlmac@newcommand@iii@opt| and |\mlmac@newcommand@iii@no@opt| then
% invoke |\mlmac@haveargs|.
%    \begin{macrocode}
\def\mlmac@newcommand@iii@opt[#1]{%
  \def\mlmac@defarg{#1}%
  \mlmac@haveargs
}
%    \end{macrocode}
%    \begin{macrocode}
\def\mlmac@newcommand@iii@no@opt{%
  \let\mlmac@defarg=\relax
  \mlmac@haveargs
}
%    \end{macrocode}
% \end{macro}
% \end{macro}
% \end{macro}
%
% \begin{macro}{\mlmac@ocamlcode}
% \begin{macro}{\mlmac@haveargs}
% Now things start to get tricky.  We have all of the arguments we need
% to define the user's command so all that's left is to grab the macro
% body.  But there's a catch: Valid OCaml code is unlikely to be valid
% \LaTeX{} code.  We therefore have to read the macro body in a
% |\verb|-like mode.  Furthermore, we actually need to \emph{store} the
% macro body in a variable, as we don't need it right away.
%
% The approach we take in |\mlmac@haveargs| is as follows.  First, we
% give all ``special'' characters category code~12 (``other'').  We then
% indicate that the carriage return character (control-M) marks the end
% of a line and that curly braces retain their normal meaning.  With the
% aforementioned category-code definitions, we now have to store the
% next curly-brace-delimited fragment of text, end the current group to
% reset all category codes to their previous value, and continue
% processing the user's macro definition.  How do we do that?  The
% answer is to assign the upcoming text fragment to a token register
% (|\mlmac@ocamlcode|) while an |\afterassignment| is in effect.  The
% |\afterassignment| causes control to transfer to |\mlmac@havecode|
% right after |\mlmac@ocamlcode| receives the macro body with all of the
% ``special'' characters made impotent.
%    \begin{macrocode}
\newtoks\mlmac@ocamlcode
%    \end{macrocode}
%    \begin{macrocode}
\def\mlmac@haveargs{%
  \begingroup
    \let\do\@makeother\dospecials
    \catcode`\^^M=\active
    \newlinechar`\^^M
    \endlinechar=`\^^M
    \catcode`\{=1
    \catcode`\}=2
    \afterassignment\mlmac@havecode
    \global\mlmac@ocamlcode
}
%    \end{macrocode}
% \end{macro}
% \end{macro}
%
% Control is transfered to |\mlmac@havecode| from |\mlmac@haveargs|
% right after the user's macro body is assigned to |\mlmac@ocamlcode|.
% We now have everything we need to define the user's macro.  The goal
% is to define it as ``|\mlmac@write@ocaml{|\meta{contents of
% Figure~\ref{fig:tofile-use}}|}|''.  This is easier said than done
% because the number of arguments in the user's macro is not known
% statically, yet we need to iterate over however many arguments there
% are.  Because of this complexity, we will explain |\mlmac@ocamlcode|
% piece-by-piece.
%
% \begin{macro}{\mlmac@sep}
% Define a character to separate each of the items presented in
% Figures~\ref{fig:tofile-define} and~\ref{fig:tofile-use}.  OCaml will need to
% strip this off each argument.  We define |\mlmac@sep| as a carriage-return
% character of category code~11 (``letter'').
%    \begin{macrocode}
{\catcode`\^^M=11\gdef\mlmac@sep{^^M}}
%    \end{macrocode}
% \end{macro}
%
% \begin{macro}{\mlmac@argnum}
% Define a loop variable that will iterate from |1| to the number of
% arguments in the user's function, i.e.,~|\mlmac@numargs|.
%    \begin{macrocode}
\newcount\mlmac@argnum
%    \end{macrocode}
% \end{macro}
%
% \begin{macro}{\mlmac@havecode}
% Now comes the final piece of what started as a call to
% |\ocaml|[|re|]|newcommand|.  First, to reset all category codes back to
% normal, |\mlmac@havecode| ends the group that was begun in
% |\mlmac@haveargs|.
%    \begin{macrocode}
\def\mlmac@havecode{%
  \endgroup
%    \end{macrocode}
%
% \begin{macro}{\mlmac@define@sub}
% We invoke |\mlmac@write@ocaml| to define a OCaml subroutine named after
% |\mlmac@cleaned@macname|.  |\mlmac@define@sub| sends OCaml the
% information shown in Figure~\vref{fig:tofile-define}.
%    \begin{macrocode}
  \edef\mlmac@define@sub{%
    \noexpand\mlmac@write@ocaml{DEF\mlmac@sep
      \mlmac@tag\mlmac@sep
      \mlmac@cleaned@macname\mlmac@sep
      \mlmac@tag\mlmac@sep
      \mlmac@argsstr\mlmac@sep
      \mlmac@tag\mlmac@sep
      \the\mlmac@ocamlcode
    }%
  }%
  \mlmac@define@sub
%    \end{macrocode}
% \end{macro}
%
% \begin{macro}{\mlmac@body}
% The rest of |\mlmac@havecode| is preparation for defining the user's
% macro.  (\LaTeXe's |\newcommand| or |\renewcommand| will do the actual
% work, though.)  |\mlmac@body| will eventually contain the complete
% (\LaTeX) body of the user's macro.  Here, we initialize it to the
% first three items listed in Figure~\vref{fig:tofile-use} (with
% intervening |\mlmac@sep|s).
%    \begin{macrocode}
  \edef\mlmac@body{%
    USE\mlmac@sep
    \mlmac@tag\mlmac@sep
    \mlmac@cleaned@macname
  }%
%    \end{macrocode}
%
% \begin{macro}{\mlmac@hash}
% Now, for each argument |#1|, |#2|,~\dots, |#\mlmac@numargs| we append
% a |\mlmac@tag| plus the argument to |\mlmac@body| (as always, with a
% |\mlmac@sep| after each item).  This requires more trickery, as \TeX{}
% requires a macro-parameter character (``|#|'') to be followed by a
% literal number, not a variable.  The approach we take, which I first
% discovered in the Texinfo source code (although it's used by \LaTeX{}
% and probably other \TeX-based systems as well), is to |\let|-bind
% |\mlmac@hash| to |\relax|.  This makes |\mlmac@hash| unexpandable, and
% because it's not a ``|#|'', \TeX{} doesn't complain.  After
% |\mlmac@body| has been extended to include |\mlmac@hash1|,
% |\mlmac@hash2|,~\dots, |\mlmac@hash\mlmac@numargs|, we then
% |\let|-bind |\mlmac@hash| to |##|, which \TeX{} lets us do because
% we're within a macro definition (|\mlmac@havecode|).  |\mlmac@body|
% will then contain |#1|, |#2|,~\dots, |#\mlmac@numargs|, as desired.
%    \begin{macrocode}
  \let\mlmac@hash=\relax
  \mlmac@argnum=1%
  \loop
    \ifnum\mlmac@numargs<\mlmac@argnum
    \else
      \edef\mlmac@body{%
        \mlmac@body\mlmac@sep\mlmac@tag%
        \mlmac@hash\mlmac@hash\number\mlmac@argnum}%
      \advance\mlmac@argnum by 1%
  \repeat
  \let\mlmac@hash=##\relax
%    \end{macrocode}
%
% \begin{macro}{\mlmac@define@command}
% We're ready to execute a |\|[|re|]|newcommand|.  Because we need to
% expand many of our variables, we |\edef| |\mlmac@define@command| to
% the appropriate |\|[|re|]|newcommand| call, which we will soon
% execute.  The user's macro must first be |\let|-bound to |\relax| to
% prevent it from expanding.  Then, we handle two cases: either all
% arguments are mandatory (and |\mlmac@defarg| is |\relax|) or the
% user's macro has an optional argument (with default value
% |\mlmac@defarg|).
%    \begin{macrocode}
  \expandafter\let\mlmac@macname=\relax
  \ifx\mlmac@defarg\relax
    \edef\mlmac@define@command{%
      \noexpand\mlmac@command\mlmac@starchar{\mlmac@macname}%
      [\mlmac@numargs]{%
        \noexpand\mlmac@write@ocaml{\mlmac@body}%
      }%
  }%
  \else
    \edef\mlmac@define@command{%
      \noexpand\mlmac@command\mlmac@starchar{\mlmac@macname}%
      [\mlmac@numargs][\mlmac@defarg]{%
        \noexpand\mlmac@write@ocaml{\mlmac@body}%
      }%
  }%
  \fi
%    \end{macrocode}
%
% The final steps are to restore the previous definition of the user's
% macro---we had set it to |\relax| above to make the name
% unexpandable---then redefine it by invoking |\mlmac@define@command|.
% Why do we need to restore the previous definition if we're just going
% to redefine it?  Because |\newcommand| needs to produce an error if
% the macro was previously defined and |\renewcommand| needs to produce
% an error if the macro was \emph{not} previously defined.
%
% |\mlmac@havecode| concludes by invoking |\mlmac@next|, which is a
% no-op for |\ocamlnewcommand| and |\ocamlrenewcommand| but processes the
% end-environment code for |\ocamlnewenvironment| and
% |\ocamlrenewenvironment|.
%    \begin{macrocode}
  \expandafter\let\mlmac@macname=\mlmac@oldbody
  \mlmac@define@command
  \mlmac@next
}
%    \end{macrocode}
% \end{macro}
% \end{macro}
% \end{macro}
% \end{macro}
%
% \subsubsection{Executing top-level OCaml expressions}
%
% \begin{macro}{\ocamlexec}
% The code for executing top-level is very similar to that used to parse OCaml
% macro bodies and for invoking OCaml macros.  Figure~\ref{fig:tofile-exec}
% illustrates the data |\mlmac@write@ocaml| passes to \ocamltex.
%
% \begin{figure}[t]
%   \hrule
%   \vspace{0.5em}
%   \centering
%   \DeleteShortVerb\|
%   \begin{tabular}{|l|}
%     \hline
%     \verb|EXEC|                   \\ \hline
%     \verb|\mlmac@tag|             \\ \hline
%     \verb|\mlmac@ocamlcode|       \\ \hline
%   \end{tabular}
%   \MakeShortVerb\|
%   \caption{Data written to \texttt{\string\mlmac@tofile} to execute a
%     top-level OCaml expression}
%   \label{fig:tofile-exec}
%   \vspace{0.5em}
%   \hrule
% \end{figure}
%    \begin{macrocode}
\def\ocamlexec{%
  \let\mlmac@next=\relax
  \begingroup
    \let\do\@makeother\dospecials
    \catcode`\^^M=\active
    \newlinechar`\^^M
    \endlinechar=`\^^M
    \catcode`\{=1
    \catcode`\}=2
    \afterassignment\mlmac@exec@havecode
    \global\mlmac@ocamlcode
}
{\catcode`\^^M=11\gdef\mlmac@sep{^^M}}
\def\mlmac@exec@havecode{%
  \endgroup%
  \edef\mlmac@exec{%
    \noexpand\mlmac@write@ocaml{EXEC\mlmac@sep
      \mlmac@tag\mlmac@sep
      \the\mlmac@ocamlcode
    }%
  }%
  \mlmac@exec
  \mlmac@next
}
%    \end{macrocode}
% \end{macro}
%
% \subsubsection{Defining OCaml environments}
% \label{sec:ocaml-env}
%
% Section~\ref{sec:ocaml-mac} detailed the implementation of
% |\ocamlnewcommand| and |\ocamlrenewcommand|.  Section~\ref{sec:ocaml-env}
% does likewise for \OCamlTeX's remaining two macros,
% |\ocamlnewenvironment| and |\ocamlrenewenvironment|, which are the
% OCaml-bodied analogues of |\newenvironment| and |\renewenvironment|.
% This section is significantly shorter than the previous because
% |\ocamlnewenvironment| and |\ocamlrenewenvironment| are largely built
% atop the macros already defined in Section~\ref{sec:ocaml-mac}.
%
% \begin{macro}{\ocamlnewenvironment}
% \begin{macro}{\ocamlrenewenvironment}
% \begin{macro}{\mlmac@command}
% \begin{macro}{\mlmac@next}
% |\ocamlnewenvironment| and |\ocamlrenewenvironment| are the remaining
% two commands exported to the user by \ocamlmac.  |\ocamlnewenvironment|
% is analogous to |\newenvironment| except that the macro body consists
% of OCaml code instead of \LaTeX{} code.  Likewise,
% |\ocamlrenewenvironment| is analogous to |\renewenvironment| except
% that the macro body consists of OCaml code instead of \LaTeX{} code.
% |\ocamlnewenvironment| and |\ocamlrenewenvironment| merely define
% |\mlmac@command| and |\mlmac@next| and invoke
% |\mlmac@newenvironment@i|.
%
% The significance of |\mlmac@next| (which was let-bound to |\relax| for
% |\ocaml|[|re|]|newcommand| but is let-bound to |\mlmac@end@environment|
% here) is that a \LaTeX{} environment definition is really two macro
% definitions: |\|\meta{name} and |\end|\meta{name}.  Because we want to
% reuse as much code as possible the idea is to define the ``begin''
% code as one macro, then inject---by way of |mlmac@next|---a call to
% |\mlmac@end@environment|, which defines the ``end'' code as a second
% macro.
%    \begin{macrocode}
\def\ocamlnewenvironment{%
  \let\mlmac@command=\newcommand
  \let\mlmac@next=\mlmac@end@environment
  % Disable optional arguments for now 
  % \@ifnextchar*{\mlmac@newenvironment@i}{\mlmac@newenvironment@i!}%
  \mlmac@newenvironment@i!%
}
%    \end{macrocode}
%    \begin{macrocode}
\def\ocamlrenewenvironment{%
  \let\mlmac@command=\renewcommand
  \let\mlmac@next=\mlmac@end@environment
  % Disable optional arguments for now
  % \@ifnextchar*{\mlmac@newenvironment@i}{\mlmac@newenvironment@i!}%
  \mlmac@newenvironment@i!%
}
%    \end{macrocode}
% \end{macro}
% \end{macro}
% \end{macro}
% \end{macro}
%
% \begin{macro}{\mlmac@newenvironment@i}
% \begin{macro}{\mlmac@starchar}
% \begin{macro}{\mlmac@envname}
% \begin{macro}{\mlmac@macname}
% \begin{macro}{\mlmac@oldbody}
% \begin{macro}{\mlmac@cleaned@macname}
% The |\mlmac@newenvironment@i| macro is analogous to
% |\mlmac@newcommand@i|; see the description of
% |\mlmac@newcommand@i|~\pageref{page:newcommand-i} to understand the
% basic structure.  The primary difference is that the environment name
% (|#2|) is just text, not a control sequence.  We store this text in
% |\mlmac@envname| to facilitate generating the names of the two macros
% that constitute an environment definition.  Note that there is no
% |\mlmac@newenvironment@ii|; control passes instead to
% |\mlmac@newcommand@ii|.
%    \begin{macrocode}
\def\mlmac@newenvironment@i#1#2{%
  \ifx#1*%
    \def\mlmac@starchar{*}%
  \else
    \def\mlmac@starchar{}%
  \fi
  \def\mlmac@envname{#2}%
  \expandafter\def\expandafter\mlmac@macname\expandafter{\csname#2\endcsname}%
  \expandafter\let\expandafter\mlmac@oldbody\mlmac@macname\relax
  \expandafter\def\expandafter\mlmac@cleaned@macname\expandafter{%
    \expandafter\string\mlmac@macname}%
  \@ifnextchar[{\mlmac@newcommand@ii}{\mlmac@newcommand@ii[0]}%]
}
%    \end{macrocode}
% \end{macro}
% \end{macro}
% \end{macro}
% \end{macro}
% \end{macro}
% \end{macro}
%
% \begin{macro}{\mlmac@end@environment}
% \begin{macro}{\mlmac@next}
% \begin{macro}{\mlmac@macname}
% \begin{macro}{\mlmac@oldbody}
% \begin{macro}{\mlmac@cleaned@macname}
% Recall that an environment definition is a shortcut for two macro
% definitions: |\|\meta{name} and |\end|\meta{name} (where \meta{name}
% was stored in |\mlmac@envname| by |\mlmac@newenvironment@i|).  After
% defining |\|\meta{name}, |\mlmac@havecode| transfers control to
% |\mlmac@end@environment| because |\mlmac@next| was let-bound to
% |\mlmac@end@environment| in |\ocaml|[|re|]|newenvironment|.
%
% |\mlmac@end@environment|'s purpose is to define |\end|\meta{name}.
% This is a little tricky, however, because \LaTeX's
% |\|[|re|]|newcommand| refuses to (re)define a macro whose name begins
% with ``|end|''.  The solution that |\mlmac@end@environment| takes is
% first to define a |\mlmac@end@macro| macro then (in |mlmac@next|)
% let-bind |\end|\meta{name} to it.  Other than that,
% |\mlmac@end@environment| is a combined and simplified version of
% |\ocamlnewenvironment|, |\ocamlrenewenvironment|, and
% |\mlmac@newenvironment@i|.
%    \begin{macrocode}
\def\mlmac@end@environment{%
  \expandafter\def\expandafter\mlmac@next\expandafter{\expandafter
    \let\csname end\mlmac@envname\endcsname=\mlmac@end@macro
    \let\mlmac@next=\relax
  }%
  \def\mlmac@macname{\mlmac@end@macro}%
  \expandafter\let\expandafter\mlmac@oldbody\csname end\mlmac@envname\endcsname
  \expandafter\def\expandafter\mlmac@cleaned@macname\expandafter{%
    \expandafter\string\mlmac@macname}%
  \@ifnextchar[{\mlmac@newcommand@ii}{\mlmac@newcommand@ii[0]}%]
}
%    \end{macrocode}
% \end{macro}
% \end{macro}
% \end{macro}
% \end{macro}
% \end{macro}
%
%
% \subsubsection{Communication between \LaTeX{} and OCaml}
%
% As shown in the previous section, when a document invokes
% |\ocaml|[|re|]|newcommand| to define a macro, \ocamlmac defines the
% macro in terms of a call to |\mlmac@write@ocaml|.  In this section, we
% learn how |\mlmac@write@ocaml| operates.
%
% At the highest level, \LaTeX-to-OCaml communication is performed via
% the filesystem.  In essence, \LaTeX{} writes a file (|\mlmac@tofile|)
% corresponding to the information in either
% Figure~\ref{fig:tofile-define} or Figure~\ref{fig:tofile-use}; OCaml
% reads the file, executes the code within it, and writes a |.tex| file
% (|\mlmac@fromfile|); and, finally, \LaTeX{} reads and executes the new
% |.tex| file.  However, the actual communication protocol is a bit more
% involved than that.  The problem is that OCaml needs to know when
% \LaTeX{} has finished writing OCaml code and \LaTeX{} needs to know
% when OCaml has finished writing \LaTeX{} code.  The solution involves
% introducing three extra files---|\mlmac@toflag|, |\mlmac@fromflag|,
% and |\mlmac@doneflag|---which are used exclusively for \LaTeX-to-OCaml
% synchronization.
%
% There's a catch: Although OCaml can create and delete files, \LaTeX{}
% can only create them.  Even worse, \LaTeX{} (more specifically,
% te\TeX, which is the \TeX{} distribution under which I developed
% \OCamlTeX) cannot reliably poll for a file's \emph{non}existence; if a
% file is deleted in the middle of an |\immediate\openin|, |latex|
% aborts with an error message.  These restrictions led to the
% regrettably convoluted protocol illustrated in
% Figure~\ref{fig:comm-protocol}.  In the figure, ``Touch'' means
% ``create a zero-length file''; ``Await'' means ``wait until the file
% exists''; and, ``Read'', ``Write'', and ``Delete'' are defined as
% expected.  Assuming the filesystem performs these operations in a
% sequentially consistent order (not necessarily guaranteed on all
% filesystems, unfortunately), \OCamlTeX{} should behave as expected.
%
% \begin{figure}[htbp]
%   \centering
%   \makeatletter\setlength{\unitlength}{\baselineskip}\makeatother
%   \DeleteShortVerb\|
%   \def\toml{\texttt{\string\mlmac@tofile}}
%   \def\frml{\texttt{\string\mlmac@fromfile}}
%   \def\tfml{\texttt{\string\mlmac@toflag}}
%   \def\ffml{\texttt{\string\mlmac@fromflag}}
%   \def\dfml{\texttt{\string\mlmac@doneflag}}
%   \begin{tabular}{@{}c@{\hspace*{1cm}}|l@{~}l|c|l@{~}l|@{}}
%     \multicolumn{1}{@{}c@{\hspace*{1cm}}}{Time} &
%     \multicolumn{2}{c}{\LaTeX} &
%     \multicolumn{1}{c}{} &
%     \multicolumn{2}{c@{}}{OCaml} \\
%
%     \cline{2-3}\cline{5-6}
%     \smash{\vector(0,-1){11}}
%     & Write & \toml &               &        &       \\
%     & Touch & \tfml & $\rightarrow$ & Await  & \tfml \\
%     &       &       &               & Read   & \toml \\
%     &       &       &               & Write  & \frml \\
%     &       &       &               & Delete & \tfml \\
%     &       &       &               & Delete & \toml \\
%     &       &       &               & Delete & \dfml \\
%     & Await & \ffml & $\leftarrow$  & Touch  & \ffml \\
%     & Read  & \frml &               &        &       \\
%     & Touch & \toml & $\rightarrow$ & Await  & \toml \\
%     &       &       &               & Delete & \ffml \\
%     & Await & \dfml & $\leftarrow$  & Touch  & \dfml \\
%     \cline{2-3}\cline{5-6}
%   \end{tabular}
%   \MakeShortVerb\|
%   \caption{\LaTeX-to-OCaml communication protocol}
%   \label{fig:comm-protocol}
% \end{figure}
%
% \begin{macro}{\mlmac@await@existence}
% \begin{macro}{\ifmlmac@file@exists}
% \begin{macro}{\mlmac@file@existstrue}
% \begin{macro}{\mlmac@file@existsfalse}
% The purpose of the |\mlmac@await@existence| macro is to repeatedly
% check the existence of a given file until the file actually exists.
% For convenience, we use \LaTeXe's |\IfFileExists| macro to check the
% file and invoke |\mlmac@file@existstrue| or |\mlmac@file@existsfalse|,
% as appropriate.
%    \begin{macrocode}
\newif\ifmlmac@file@exists
%    \end{macrocode}
%    \begin{macrocode}
\newcommand{\mlmac@await@existence}[1]{%
  \loop
    \IfFileExists{#1}%
                 {\mlmac@file@existstrue}%
                 {\mlmac@file@existsfalse}%
    \ifmlmac@file@exists
    \else
  \repeat
}
%    \end{macrocode}
% \end{macro}
% \end{macro}
% \end{macro}
% \end{macro}
%
% \begin{macro}{\mlmac@outfile}
% We define a file handle for |\mlmac@write@ocaml@i| to use to create and
% write |\mlmac@tofile| and |\mlmac@toflag|.
%    \begin{macrocode}
\newwrite\mlmac@outfile
%    \end{macrocode}
% \end{macro}
%
% \begin{macro}{\mlmac@write@ocaml}
% |\mlmac@write@ocaml| begins the \LaTeX-to-OCaml data exchange, following
% the protocol illustrated in Figure~\ref{fig:comm-protocol}.
% |\mlmac@write@ocaml| prepares for the next piece of text in the input
% stream to be read with ``special'' characters marked as category
% code~12 (``other'').  This prevents \LaTeX{} from complaining if the
% OCaml code contains invalid \LaTeX{} (which it usually will).
% |\mlmac@write@ocaml| ends by passing control to |\mlmac@write@ocaml@i|,
% which performs the bulk of the work.
%    \begin{macrocode}
\newcommand{\mlmac@write@ocaml}{%
  \begingroup
    \let\do\@makeother\dospecials
    \catcode`\^^M=\active
    \newlinechar`\^^M
    \endlinechar=`\^^M
    \catcode`\{=1
    \catcode`\}=2
    \mlmac@write@ocaml@i
}
%    \end{macrocode}
% \end{macro}
%
% \begin{macro}{\mlmac@write@ocaml@i}
% When |\mlmac@write@ocaml@i| begins executing, the category codes are
% set up so that the macro's argument will be evaluated ``verbatim''
% except for the part consisting of the \LaTeX\ code passed in by the
% author, which is partially expanded.  Thus, everything is in place for
% |\mlmac@write@ocaml@i| to send its argument to Perl and read back the
% (\LaTeX) result.
%
% Because\label{page:define-dummies} all of \ocamlmac's protocol
% processing is encapsulated within |\mlmac@write@ocaml@i|, this is the
% only macro that strictly requires \ocamltex.  Consequently, we wrap the
% entire macro definition within a check for \ocamltex.
%    \begin{macrocode}
\ifmlmac@have@ocamltex
  \newcommand{\mlmac@write@ocaml@i}[1]{%
%    \end{macrocode}
% The first step is to write argument |#1| to |\mlmac@tofile|:
%    \begin{macrocode}
      \immediate\openout\mlmac@outfile=\mlmac@tofile\relax
      \let\protect=\noexpand
      \def\begin{\noexpand\begin}%
      \def\end{\noexpand\end}%
      \immediate\write\mlmac@outfile{#1}%
      \immediate\closeout\mlmac@outfile
%    \end{macrocode}
% \noindent
% (In the future, it might be worth redefining |\def|, |\edef|, |\gdef|,
% |\xdef|, |\let|, and maybe some other control sequences as
% ``|\noexpand|\meta{control sequence}|\noexpand|'' so that |\write|
% doesn't try to expand an undefined control sequence.)
%
% We're now finished using |#1| so we can end the group begun by
% |\mlmac@write@ocaml|, thereby resetting each character's category code
% back to its previous value.
%    \begin{macrocode}
    \endgroup
%    \end{macrocode}
%
% Continuing the protocol illustrated in Figure~\ref{fig:comm-protocol},
% we create a zero-byte |\mlmac@toflag| in order to notify \ocamltex
% that it's now safe to read |\mlmac@tofile|.
%    \begin{macrocode}
    \immediate\openout\mlmac@outfile=\mlmac@toflag\relax
    \immediate\closeout\mlmac@outfile
%    \end{macrocode}
%
% To avoid reading |\mlmac@fromfile| before \ocamltex has finished
% writing it we must wait until \ocamltex creates |\mlmac@fromflag|,
% which it does only after it has written |\mlmac@fromfile|.
%    \begin{macrocode}
    \mlmac@await@existence\mlmac@fromflag
%    \end{macrocode}
%
% At this point, |\mlmac@fromfile| should contain valid \LaTeX{} code.
% However, we defer inputting it until we the very end.  Doing so
% enables recursive and mutually recursive invocations of \OCamlTeX
% macros.
%
% Because \TeX{} can't delete files we require an additional
% \LaTeX-to-OCaml synchronization step.  For convenience, we recycle
% |\mlmac@tofile| as a synchronization file rather than introduce yet
% another flag file to complement |\mlmac@toflag|, |\mlmac@fromflag|,
% and |\mlmac@doneflag|.
%    \begin{macrocode}
    \immediate\openout\mlmac@outfile=\mlmac@tofile\relax
    \immediate\closeout\mlmac@outfile
    \mlmac@await@existence\mlmac@doneflag
%    \end{macrocode}
%
% The only thing left to do is to |\input| and evaluate
% |\mlmac@fromfile|, which contains the \LaTeX{} output from the OCaml
% function.
%    \begin{macrocode}
    \input\mlmac@fromfile\relax
  }
%    \end{macrocode}
%
% The foregoing code represents the ``real'' definition of
% |\mlmac@write@ocaml@i|.  For the user's convenience, we define a dummy
% version of |\mlmac@write@ocaml@i| so that a document which utilizes
% \ocamlmac can still compile even if not built using \ocamltex.  All
% calls to macros defined with |\ocaml|[|re|]|newcommand| and all
% invocations of environments defined with |\ocaml|[|re|]|newenvironment|
% are replaced with ``\fbox{OCaml\TeX}''.  A minor complication is that
% text can't be inserted before the |\begin{document}|.  Hence, we
% initially define |\mlmac@write@ocaml@i| as a do-nothing macro and
% redefine it as ``|\fbox{OCaml\TeX}|'' at the |\begin{document}|.
%    \begin{macrocode}
\else
  \newcommand{\mlmac@write@ocaml@i}[1]{\endgroup}
  \AtBeginDocument{%
    \renewcommand{\mlmac@write@ocaml@i}[1]{%
%    \end{macrocode}
% \begin{macro}{\mlmac@show@placeholder}
% There's really no point in outputting a framed ``OCaml\TeX'' when a
% macro is defined \emph{and} when it's used.  |\mlmac@show@placeholder|
% checks the first character of the protocol header.  If it's
% ``D''~(|DEF|), nothing is output.  Otherwise, it'll be ``U''~(|USE|)
% and ``OCaml\TeX'' will be output.
%    \begin{macrocode}
      \def\mlmac@show@placeholder##1##2\@empty{%
        \ifx##1D\relax
          \endgroup
        \else
          \endgroup
          \fbox{OCaml\TeX}%
        \fi
      }%
      \mlmac@show@placeholder#1\@empty
    }%
  }
\fi
%    \end{macrocode}
% \end{macro}
% \end{macro}
%
% \iffalse
%</package>
% \fi
%
% ^^A  Keep track of the number of lines of code in ocamltex.
% \makeatletter
% \immediate\write\@auxout{^^A
%   \noexpand\gdef\noexpand\ocamlmaclines{\the\c@CodelineNo}}
% \makeatother
%
%
% \subsection{\ocamltex}
% \label{sec:ocamltex}
%
% \ocamltex is a wrapper script for |latex| (or any other \LaTeX{}
% compiler).  It sets up client-server communication between \LaTeX{}
% and OCaml, with \LaTeX{} as the client and OCaml as the server.  When a
% \LaTeX{} document sends a piece of OCaml code to \ocamltex (with the
% help of \ocamlmac, as detailed in Section~\ref{sec:ocamlmacros}),
% \ocamltex executes it within a secure sandbox and transmits the
% resulting \LaTeX{} code back to the document.
%
% \iffalse
%<*ocamltex>
% \fi
%
% \subsubsection{Header comments}
%
% Because \ocamltex is generated without a \textsf{DocStrip}
% preamble or postamble we have to manually include the desired
% text as OCaml comments.
%
%
% \paragraph{Variables corresponding to command-line arguments}
% \begin{ocamlmacro}{$latexprog}
% \begin{ocamlmacro}{$runsafely}
% \begin{ocamlmacro}{@permittedops}
% |$latexprog| is the name of the \LaTeX{} executable
% (e.g.,~``|latex|'').  If |$runsafely| is~|1| (the default), then the
% user's Perl code runs in a secure sandbox; if it's |0|, then arbitrary
% Perl code is allowed to run.  |@permittedops| is a list of features
% made available to the user's Perl code.  Valid values are described in
% Perl's |Opcode| manual page.  \ocamltex's default is a list containing
% only |:browse|.
% \end{ocamlmacro}
% \end{ocamlmacro}
% \end{ocamlmacro}
%
% \paragraph{Other global variables}
% \begin{ocamlmacro}{$progname}
% \begin{ocamlmacro}{$jobname}
% \begin{ocamlmacro}{@latexcmdline}
% \begin{ocamlmacro}{$toperl}
% \begin{ocamlmacro}{$fromperl}
% \begin{ocamlmacro}{$toflag}
% \begin{ocamlmacro}{$fromflag}
% \begin{ocamlmacro}{$doneflag}
% \begin{ocamlmacro}{$logfile}
% \begin{ocamlmacro}{$sandbox}
% \begin{ocamlmacro}{$latexpid}
% |$progname| is the run-time name of the \ocamltex program.
% |$jobname| is the base name of the user's |.tex| file, which defaults
% to the \TeX{} default of |texput|.  |@latexcmdline| is the command
% line to pass to the \LaTeX{} executable.  |$toperl| defines the
% filename used for \LaTeX{}$\rightarrow$OCaml communication.
% |$fromperl| defines the filename used for Perl$\rightarrow$\LaTeX{}
% communication.  |$toflag| is the name of a file that will exist only
% after \LaTeX{} creates |$tofile|.  |$fromflag| is the name of a file
% that will exist only after Perl creates |$fromfile|.  |$doneflag| is
% the name of a file that will exist only after OCaml deletes
% |$fromflag|.  |$logfile| is the name of a log file to which \ocamltex
% writes verbose execution information.  |$sandbox| is a secure sandbox
% in which to run code that appeared in the \LaTeX{} document.
% |$latexpid| is the process~ID of the |latex| process.
% \end{ocamlmacro}
% \end{ocamlmacro}
% \end{ocamlmacro}
% \end{ocamlmacro}
% \end{ocamlmacro}
% \end{ocamlmacro}
% \end{ocamlmacro}
% \end{ocamlmacro}
% \end{ocamlmacro}
% \end{ocamlmacro}
% \end{ocamlmacro}
%
% \subsubsection{Command-line conversion}
%
% In this section, \ocamltex parses its own command line and prepares a
% command line to pass to |latex|.
%
% \paragraph{Parsing \ocamltex's command line}
% We first set |$latexprog| to be the contents of the environment
% variable |PERLTEX| or the value ``|latex|'' if |PERLTEX| is not
% specified.  We then use |Getopt::Long| to parse the command line,
% leaving any parameters we don't recognize in the argument vector
% (|@ARGV|) because these are presumably |latex| options.
%
% \paragraph{Preparing a \LaTeX{} command line}
% \begin{ocamlmacro}{$firstcmd}
% \begin{ocamlmacro}{$option}
% We start by searching |@ARGV| for the first string that does not start
% with ``|-|'' or ``|\|''.  This string, which represents a filename, is
% used to set |$jobname|.
% \end{ocamlmacro}
% \end{ocamlmacro}
%
% \begin{ocamlmacro}{$separator}
% To avoid conflicts with the code and parameters passed to Perl from
% \LaTeX{} (see Figure~\vref{fig:tofile-define} and
% Figure~\vref{fig:tofile-use}) we define a separator string,
% |$separator|, containing 20 random uppercase letters.
% \end{ocamlmacro}
%
% Now that we have the name of the \LaTeX{} job (|$jobname|) we can
% assign |$toperl|, |$fromperl|, |$toflag|, |$fromflag|, |$doneflag|,
% and |$logfile| in terms of |$jobname| plus a suitable extension.
%
% We now replace the filename of the |.tex| file passed to \ocamltex
% with a |\def|inition of the separator character, |\def|initions of the
% various files, and the original file with |\input| prepended if
% necessary.
%
% \subsubsection{Launching \LaTeX}
%
% We start by deleting the |$toperl|, |$fromperl|, |$toflag|,
% |$fromflag|, and |$doneflag| files, in case any of these were left
% over from a previous (aborted) run.  We also create a log file,
% |$logfile|.  As |@latexcmdline| contains the complete command line to
% pass to |latex| we need only |fork| a new process and have the child
% process overlay itself with |latex|.  \ocamltex continues running as
% the parent.
%
% Note that here and elsewhere in \ocamltex, |unlink| is called
% repeatedly until the file is actually deleted.  This works around a
% race condition that occurs in some filesystems in which file deletions
% are executed somewhat lazily.
%
% \subsubsection{Communicating with \LaTeX{}}
%
% The following code constitutes \ocamltex's main loop.  Until |latex|
% exits, the loop repeatedly reads Perl code from \LaTeX{}, evaluates
% it, and returns the result as per the protocol described in
% Figure~\vref{fig:comm-protocol}.
%
% \begin{ocamlmacro}{$awaitexists}
% We define a local subroutine |$awaitexists| which waits for a given
% file to exist.  If |latex| exits while |$awaitexists| is waiting, then
% \ocamltex cleans up and exits, too.
% \end{ocamlmacro}
%
% \begin{ocamlmacro}{$entirefile}
% Wait for |$toflag| to exist.  When it does, this implies that
% |$toperl| must exist as well.  We read the entire contents of
% |$toperl| into the |$entirefile| variable and process it.
% Figures~\ref{fig:tofile-define} and~\ref{fig:tofile-use} illustrate
% the contents of |$toperl|.
% \end{ocamlmacro}
%
% \begin{ocamlmacro}{$optag}
% \begin{ocamlmacro}{$macroname}
% \begin{ocamlmacro}{@otherstuff}
% We split the contents of |$entirefile| into an operation tag (either
% |DEF| or |USE|), the macro name, and everything else (|@otherstuff|).
% If |$optag| is |DEF| then |@otherstuff| will contain the Perl code to
% define.  If |$optag| is |USE| then |@otherstuff| will be a list of
% subroutine arguments.
% \end{ocamlmacro}
% \end{ocamlmacro}
% \end{ocamlmacro}
%
% We clean up the macro name by deleting all leading non-letters,
% replacing all subsequent non-alphanumerics with ``|_|'', and
% prepending ``|latex_|'' to the macro name.
%
% If we're calling a subroutine, then we make the arguments more
% palatable to Perl by single-quoting them and replacing every
% occurrence of ``|\|'' with ``|\\|'' and every occurrence of ``|'|''
% with ``|\'|''.
%
% \begin{ocamlmacro}{$perlcode}
% There are two possible values that can be assigned to |$perlcode|.  If
% |$optag| is |DEF|, then |$perlcode| is made to contain a definition of
% the user's subroutine, named |$macroname|.  If |$optag| is |USE|, then
% |$perlcode| becomes an invocation of |$macroname| which gets passed
% all of the macro arguments.  Figure~\ref{fig:latex-perl-define}
% presents an example of how the following code converts a \PerlTeX{}
% macro definition into a Perl subroutine definition and
% Figure~\ref{fig:latex-perl-use} presents an example of how the
% following code converts a \PerlTeX{} macro invocation into a Perl
% subroutine invocation.
%
% \begin{figure}[tbp]
%   \centering
%   \DeleteShortVerb\|
%   \begin{tabular}{l|l|}
%     \cline{2-2}
%     \LaTeX: & \verb|\ocamlnewcommand{\mymacro}[2]{%| \\
%             & \verb|  sprintf "Isn't $_[0] %s $_[1]?\n",| \\
%             & \verb|    $_[0]>=$_[1] ? ">=" : "<"| \\
%             & \verb|}| \\
%     \cline{2-2}
%     \multicolumn{2}{c}{} \\
%     \multicolumn{1}{c}{} & \multicolumn{1}{c}{\Huge$\Downarrow$} \\
%     \multicolumn{2}{c}{} \\
%     \cline{2-2}
%     OCaml:   & \verb|sub latex_mymacro {| \\
%             & \verb|  sprintf "Isn't $_[0] %s $_[1]?\n",| \\
%             & \verb|    $_[0]>=$_[1] ? ">=" : "<"| \\
%             & \verb|}| \\
%     \cline{2-2}
%   \end{tabular}
%   \MakeShortVerb\|
%   \caption{Conversion from \LaTeX{} to OCaml (subroutine definition)}
%   \label{fig:latex-ocaml-define}
% \end{figure}
%
% \begin{figure}[tbp]
%   \centering
%   \DeleteShortVerb\|
%   \begin{tabular}{l|l|}
%     \cline{2-2}
%     \LaTeX: & \verb|\mymacro{12}{34}| \\
%     \cline{2-2}
%     \multicolumn{2}{c}{} \\
%     \multicolumn{1}{c}{} & \multicolumn{1}{c}{\Huge$\Downarrow$} \\
%     \multicolumn{2}{c}{} \\
%     \cline{2-2}
%     Perl:   & \verb|latex_mymacro ('12', '34');| \\
%     \cline{2-2}
%   \end{tabular}
%   \MakeShortVerb\|
%   \caption{Conversion from \LaTeX{} to Perl (subroutine invocation)}
%   \label{fig:latex-perl-use}
% \end{figure}
%
% \end{ocamlmacro}
%
% Log what we're about to evaluate.
%
% \begin{ocamlmacro}{$result}
% \begin{ocamlmacro}{$msg}
% We're now ready to execute the user's code using the |$sandbox->reval|
% function.  If a warning occurs we write it as a Perl comment to the
% log file.  If an error occurs (i.e.,~|$@| is defined) we replace the
% result (|$result|) with a call to \LaTeXe's |\PackageError| macro to
% return a suitable error message.  We produce one error message for
% sandbox policy violations (detected by the error message, |$@|,
% containing the string ``\texttt{trapped by}'') and a different error
% message for all other errors caused by executing the user's code.  For
% clarity of reading both warning and error messages, we elide the
% string ``\texttt{at (eval} \meta{number}\texttt{) line}
% \meta{number}''.
% \end{ocamlmacro}
% \end{ocamlmacro}
%
% Log the resulting \LaTeX{} code.
%
% We add |\endinput| to the generated \LaTeX{} code to suppress an
% extraneous end-of-line character that \TeX{} would otherwise insert.
%
% Continuing the protocol described in Figure~\ref{fig:comm-protocol}
% we now write |$result| (which contains either the result of executing
% the user's or a |\PackageError|) to the |$fromperl| file, delete
% |$toflag|, |$toperl|, and |$doneflag|, and notify \LaTeX{} by touching
% the |$fromflag| file.
%
% We have to perform one final \LaTeX-to-OCaml synchronization step.
% Otherwise, a subsequent |\ocaml|[|re|]|newcommand| would see that
% |$fromflag| already exists and race ahead, finding that |$fromperl|
% does not contain what it's supposed to.
%
% \subsubsection{Final cleanup}
%
% If we exit abnormally we should do our best to kill the child |latex|
% process so that it doesn't continue running forever, holding onto
% system resources.
%
%    \begin{macrocode}
#!/usr/bin/env cash
(*
# vim:ft=ocaml 
*)

(* Prepare a LaTeX run for two-way communication with OCaml
   By Geoffrey Alan Washburn <washburn@acm.org> based upon
   code by Scott Pakin <scott+pt@pakin.org>                 *)

(* 
 This is file `ocamltex', Version 0.5
 
 Copyright (C) 2006 by Geoffrey Alan Washburn <washburn@acm.org>
 Copyright (C) 2004 by Scott Pakin <scott+pt@pakin.org>

 This file may be distributed and/or modified under the conditions
 of the LaTeX Project Public License, either version 1.2 of this
 license or (at your option) any later version.  The latest
 version of this license is in:

    http://www.latex-project.org/lppl.txt

 and version 1.2 or later is part of all distributions of LaTeX
 version 1999/12/01 or later. *)

(* Initialize the top-level we are going to interact with *) 
let () = Toploop.initialize_toplevel_env ();;

(* Load the regular expression library *)
#load "str.cma";;
open Cash;;
open Str;;

(********************************************************************)

(* Set the default "program name" *)
let progname = ref "ocamltex"

(* Set the default "jobname" *)
let jobname = ref "texput" 

(* Display the "usage" information when run? *)
let showUsage = ref false 

(* Set the default file names *)
let toocaml = ref ((!jobname) ^ ".toml")
let fromocaml = ref ((!jobname) ^ ".frml")
let toflag = ref ((!jobname) ^ ".tfml")
let fromflag = ref ((!jobname) ^ ".ffml")
let doneflag = ref ((!jobname) ^ ".dfml") 
let logfile = ref ((!jobname) ^ ".lgml")

(* Create a string reference used to communicate results from "eval" *)
let evalData = ref ""

(* By default use "latex" *)
let latexprog = try ref (getenv "OCAMLTEX") with _ -> ref "latex" 

(* Separator character for directory names *)
let dirsep = "/"

(********************************************************************)

(* Print an error message and exit with an error code *)
let error (msg : string) : 'a = 
  begin
    print_string ((!progname) ^ " (error) : " ^ msg ^ "\n"); 
    exit 1
  end  

(* basename *)
let basename (fn : string) =
  let parts = split (regexp dirsep) fn in
  let rec last (l : string list) : string = 
    (match l with 
     | []     -> (error "Argument to basename doesn't have a basecase?!")
     | [base] -> base
     | h::t   -> last t) in
   last (parts)

(* If a file exists, delete it *)
let delete_file_exists f = if (is_file_existing_fn f) then delete_file f

(* Remove trailing newline from a string *)
let chomp s = 
    if s.[(String.length s) - 1] = '\n' then
      String.sub s 0 ((String.length s) - 1)
    else
      s

(* Evaluate the supplied string as a OCaml top-level input *)
let eval txt = let lb = (Lexing.from_string txt) in
 let phr = !Toploop.parse_toplevel_phrase lb in
 Toploop.execute_phrase true Format.str_formatter phr 

(* Remove .tex suffix *)
let remove_tex_suffix file = global_replace (regexp "\\.tex$") "" file

(********************************************************************)

(* Get the name of the program that was run. *)
let () = progname := basename (List.hd (command_line ()))

(* Set the name of the files used *)
let namefiles () =
    begin
      toocaml := (!jobname) ^ ".toml";
      fromocaml := (!jobname) ^ ".frml";
      toflag := (!jobname) ^ ".tfml";
      fromflag := (!jobname) ^ ".ffml";
      doneflag := (!jobname) ^ ".dfml";
      logfile := (!jobname) ^ ".lgml";
    end  

(* Create the separator string.

   Can probably use something shorter than 20, because probability of collision 
   with 1/26^20 is really low.*)
let () = Random.self_init ()
let rec genSeparator n = 
    match n with
        | 0 -> ""
        | n -> (Char.escaped (Char.chr ((Char.code 'A') + (Random.int 26)))) ^ genSeparator (n - 1) 
let separator = genSeparator 20


(* Remove temporary files, if they exist *)
let filelist () = [!toocaml; !fromocaml; !toflag; !fromflag; !doneflag]
let cleanup () = ignore (List.map delete_file_exists (filelist ()))
         
(* Prelude information sent to TeX *)
let firstcmd () = 
    "\\makeatletter" ^ 
    "\\def\\mlmac@pid{0}" ^ (* FIX! *)
    "\\def\\mlmac@tag{" ^ separator ^ "}" ^    
    "\\def\\mlmac@tofile{" ^ (!toocaml) ^ "}" ^
    "\\def\\mlmac@fromfile{" ^ (!fromocaml) ^ "}" ^
    "\\def\\mlmac@toflag{" ^ (!toflag) ^ "}" ^ 
    "\\def\\mlmac@fromflag{" ^ (!fromflag) ^ "}" ^
    "\\def\\mlmac@doneflag{" ^ (!doneflag) ^ "}" ^ 
    "\\makeatother" ^
    "\\input{" ^ (!jobname) ^ "}"

(* Parses the command line arguments, handling the ones it understands,
   assumes the rest are file names to be processed *)
let rec parseArgs (args : string list) : string list =
  (match args with
   | []               -> []
   | "-h"::t          -> (showUsage := true); parseArgs t
   | "--h"::t         -> (showUsage := true); parseArgs t
   | "--help"::t      -> (showUsage := true); parseArgs t
   | (h::t)            
     when 
     (string_match (regexp "--latex=\\(.*\\)") h 0) -> 
      begin
        latexprog := (matched_group 1 h); 
        parseArgs t
      end
   | h::t            -> h :: (parseArgs t))

exception FileDone

(* Loop until the specified file exists, or the specified process exits *)
let rec awaitexists (file : string) pid =
    if (not (is_file_existing_fn file)) then 
      begin
        sleep 0;
        try (match (wait ~wflags:[WNOHANG] pid) with 
            | WEXITED pid' -> cleanup (); raise FileDone
            | WSIGNALED pid' -> ()
            | WSTOPPED pid' -> ())
        with Child_not_ready -> awaitexists file pid   
      end
    else
      ()

(* Process a message sent from TeX *)
let process_input (logCh : out_channel) (data : string) : string =  
    let chunks = List.map chomp (split (regexp separator) data) in
      match chunks with
          | optarg::tail ->
            (match optarg with
                 | ("USE"|"DEF") ->
                   (match tail with mname::tail ->
                     let mname = global_replace (regexp "^[^A-Za-z]+") "" mname in
                     let mname = global_replace (regexp "\\W") "_" mname in
                     let mname = "latex_" ^ mname in
                     (match optarg with
                          | "USE" -> 
                            let args = List.map (fun arg -> "\"" ^ arg ^ "\"") tail in
                            let args' = String.concat " " args in
                            let cmd = "let _ = evalData := " ^ mname ^ " " ^ args' ^ ";;" in
                            let () = write_string_out ~dst:logCh ("(* Using OCaml code *)\n\n" ^ cmd ^ 
                                                                "\n\n(********************)\n") in
                              ignore (eval cmd) ; !evalData
                          | "DEF" -> 
                            (match tail with
                                 | (args_string::tail') ->
                                   let args = split (regexp ",") args_string in
                                   let args' = List.map (fun arg -> "(" ^ arg ^ " : string)") args in
                                   let args'' = String.concat " " args' in
                                   let body = String.concat "" tail' in
                                   let cmd = "let rec " ^ mname ^ " " ^ args'' ^ " : string = " ^ body ^ ";;" in 
                                   let () = write_string_out ~dst:logCh 
                                            ("(* Defining OCaml function *)\n\n" ^ cmd ^ 
                                             "\n\n(***************************)\n") in
                                   ignore (eval cmd) ; ""
                                 | _ -> error ("Bad input stream:" ^ data))) 
                    | _ -> error ("Bad input stream:" ^ data))
                 | "EXEC" -> 
                    let code = String.concat "" tail in
                    let cmd = code ^ ";;" in
                    let () = write_string_out ~dst:logCh ("(* Execing OCaml code *)\n\n" ^ cmd ^ 
                                                          "\n\n(**********************)\n") in
                       ignore (eval cmd) ; ""
                 | _ -> error ("Bad opcode:" ^ data)) 
          | _ -> error ("Bad input stream:" ^ data)

(* Loop waiting for definition/evaluation requests *)
let rec main_loop (logCh : out_channel) pid =
    let () = awaitexists (!toflag) pid in
    let toocamlCh = open_input_file (!toocaml) in
    let entirefile = string_of_in_channel toocamlCh in
    let _ = close_in toocamlCh in 
    let result = process_input logCh entirefile in
    let result = result ^ "\\endinput" in
    let () = write_string_out ~dst:logCh 
                              ("(* OCaml evaluation result *)\n\n" ^ result ^ 
                               "\n\n(***************************)\n") in
    let () = delete_file_exists (!fromocaml) in
    let fromocamlCh = open_output_file ~flags:[O_CREAT] (!fromocaml) in
    let () = write_string_out ~dst:fromocamlCh result in
    let _ = close_out fromocamlCh in
    let () = delete_file_exists (!toflag) in
    let () = delete_file_exists (!toocaml) in
    let () = delete_file_exists (!doneflag) in
    let _ = close_out (open_output_file ~flags:[O_CREAT] (!fromflag)) in
    let () = awaitexists (!toocaml) pid in
    let () = delete_file_exists (!fromflag) in
    let _ = close_out (open_output_file ~flags:[O_CREAT] (!doneflag)) in
      main_loop logCh pid

(* Loop over the files supplied on the command-line *)
let rec file_loop files =
    match files with
        | [] -> exit 0
        | h::t -> 
          let () = jobname := remove_tex_suffix h in
          let () = namefiles () in
          (* Clean up temporary files that might be hanging around from a previous run. *)
          let () = cleanup () in
          let pid = (match fork () with
                   | Some(latexpid) ->  latexpid
                   | None -> exec_path (!latexprog) [firstcmd ()]) in
          let logfileCh = open_output_file ~flags:[O_CREAT] (!logfile) in
          let () = (try main_loop logfileCh pid with FileDone -> ()) in
          let _ = close_out logfileCh in
            file_loop t

(* Parse the command line and iterate over the input files *)
let main () =
      let () = print_endline 
               ("This is OCamlTeX, Version 0.5 (Copyright 2006 by Geoffrey Washburn)") in
      let args = parseArgs (List.tl (command_line ())) in
        if !showUsage then
          begin
            print_endline 
"Usage: ocamltex [-h|--h|-help|--help] [--latex=program] files...
Options:
ocamltex accepts the following command-line options

    -h,-help,--h,--help
        Display basic usage information.

    --latex=program
        Specify a program to use instead of latex. For example,
        \"--latex=pdflatex\" would typeset the given document using pdflatex
        instead of ordinary latex.
";
            exit 0
          end    
        else
          file_loop args

(* Set everything in motion *)
let _ = main () 
%    \end{macrocode}
%
% \iffalse
%</ocamltex>
% \fi
%
% ^^A  Keep track of the number of lines of code in ocamltex.
% \makeatletter
% \immediate\write\@auxout{^^A
%   \noexpand\gdef\noexpand\ocamltexlines{\the\c@CodelineNo}}
% \makeatother
%
% \subsection{Porting to other languages}
% \label{sec:porting}
%
% OCaml is a very powerful programming language. However, OCaml is not as well
% known as some other languages.  Therefore, some users will therefore long for
% a \meta{some-language-other-than-OCaml}\TeX\@.  Fortunately, porting
% \OCamlTeX to use a different language is straightforward.  \OCamlTeX itself 
% is actually derived from \PerlTeX by Scott Pakin.
%
% \ocamltex will need to be rewritten in the target language, of course. I
% found this fairly straightforward if the language provides reasonable support
% for |fork|/|exec| and file operations.
%
% Modifications to \ocamlmac are likely be fairly minimal.  In all probability,
% only the following changes will need to be made:
%
% \begin{itemize}
%   \item Rename \ocamlmac and \ocamltex (and choose a package name
%   other than ``\OCamlTeX'') as per the \OCamlTeX license agreement
%   (Section~\ref{sec:license}).
%
%   \item In your replacement for \ocamlmac, replace all occurrences of
%   ``|mlmac|'' with a different string.
%
%   \item In your replacement for \ocamltex, choose different file
%   extensions for the various helper files.
% \end{itemize}
%
% The importance of these changes is that they help ensure version consistency
% and that they make it possible to run
% \meta{some-language-other-than-OCaml}\TeX{} alongside \OCamlTeX, enabling
% multiple programming languages to be utilized in the same \LaTeX{} document.
%
% \subsection{Differences between \OCamlTeX}
% \label{sec:differences}
%
% \PerlTeX does not provide named arguments like \OCamlTeX does, instead
% arguments are pushed onto the Perl stack.  \PerlTeX does not provide anything
% equivalent to |\ocamlexec|, but it probably does not need such a capability
% because, as far as I know, any valid Perl code may occur within a subroutine. 
% \OCamlTeX does not yet provide optional argument support.
%
% Unlike \OCamlTeX, \PerlTeX executes the Perl code within a secure sandbox.
% This means that potentially harmful Perl operations, such as |unlink|,
% |rmdir|, and |system| will result in a run-time error.  Having a secure
% sandbox implies that it is safe to build \PerlTeX documents written by other
% people without worrying about what they may do to your computer system.  It
% would be great to have something like this for \OCamlTeX, but I currently am
% not only simple way to provide sandboxing without building a custom OCaml
% runtime.
%
% \subsection{Future work}
% \label{sec:future}
% 
% \OCamlTeX is still under active development and there are several parts that
% are either incomplete or would be very useful to have in future releases.
% \begin{itemize}
% \item \OCamlTeX's versions of |\ocaml|[|re|]|command| and
% |\ocaml|[|re|]|environment| currently do not support optional arguments.
% This should not require much modification.  This is probably best implemented
% using OCaml's optional argument feature.
%
% \item \ocamltex does not currently implement or make use of the |\mlmac@pid|.
%
% \item \ocamltex depends upon the Caml Shell (CASH) package.  It might be
% useful to eliminate this dependency to make \ocamltex slightly more portable.
%
% \item It would be useful to extend \ocamltex with data structures and library
% functions for generating and manipulating \TeX\ data.  A ``\TeX\ monad''
% implemented via the |pa_monad| package would be very useful for this.
%
% \end{itemize}
%
% \begin{thebibliography}{2}
%
% \bibitem{Ant}
% Achim Blumensath. 
% \newblock The Ant Typesetting System.
% \newblock \url{find the url}.
% \bibitem{OCaml}
% Xavier Leroy.
% \newblock The OCaml Programming Language.
% \newblock \url{find the url}.
% \end{thebibliography}
% \Finale
\endinput
